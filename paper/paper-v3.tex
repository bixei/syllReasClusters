\documentclass[12pt]{article}



\usepackage[margin=3cm]{geometry}


\renewcommand{\baselinestretch}{1.5} 

\usepackage[hidelinks]{hyperref} 

\usepackage{graphicx}
\usepackage{tikz}
\usetikzlibrary{shapes,positioning,arrows,backgrounds,fit,calc,automata}   
\usetikzlibrary{decorations.markings} 
\usepackage{xcolor}
\usepackage{mathtools} 
\usepackage{booktabs}  
\usepackage{xspace}
\usepackage{color}
\usepackage{amsfonts}	
\usepackage{amsthm}
\usepackage{amssymb}
\usepackage{amsmath}
\usepackage[shortlabels]{enumitem}
\usepackage{etoolbox}  
\usepackage{cleveref}   
\usepackage{wrapfig}
\usepackage{caption} 
\usepackage{adjustbox}

\usepackage[framemethod=TikZ]{mdframed}
\usepackage{amsthm}
\usepackage{thmtools}
\usepackage{amsthm,thmtools,xcolor}

\usepackage{multirow}

\usepackage{palatino}
\usepackage[normalem]{ulem}
\usepackage{color}
\usepackage[utf8x]{inputenc}
\usepackage{default}
\usepackage{xcolor}
\usepackage{colortbl}
\usepackage{amssymb}
\usepackage{amsmath}
\usepackage{array,amsmath}
\usepackage{booktabs}
\usepackage{ulem}
\usepackage{tikz}
\usepackage{graphicx,wrapfig,lipsum}
\usepackage{rotating}
\usepackage{xcolor}
\usepackage{scalerel}

\usepackage[backgroundcolor=orange!5,bordercolor=orange,linecolor=orange,textsize=footnotesize]{todonotes}


% \theoremstyle{plain}
% 
% \newtheorem{definition}{Definition}[chapter]
% 
% \newtheorem{proposition}{Proposition}[chapter]
% 
% \newtheorem{corollary}[proposition]{Corollary}
% 
% \newtheorem{theorem}[proposition]{Theorem}
% 
% \newtheorem{lemma}[proposition]{Lemma}
% 
% \newtheorem{example}{Example}[section]

\newcommand{\ud}{\mathsf{undef}}
\newcommand{\defined}{\mathsf{def}}

\newcommand{\Shrink}{\vspace{-1mm}}

\newcommand{\gcolor}{\color{gray}}


\usepackage{collcell}
\makeatletter
\newcolumntype{G}{>{\collectcell\@gobble}c<{\endcollectcell}@{}}
\makeatother

% \newcounter{theorem}
% \newcounter{subprop}[theorem]
% \renewcommand{\thesubprop}{\arabic{theorem}.\roman{subprop}}
% \newcommand{\slab}[1]
% {\noindent%
% \refstepcounter{subprop}%
% \label{#1}%
% \makebox[2.3em][r]{\textnormal{(\roman{subprop})} \hspace{0.25em}}}



\newcommand{\SvL}{\mathsf{SvL}}

\newcommand{\MP}{\CalM_\CalP}

\newcommand{\Svl}{\Phi}
\newcommand{\Svlp}{\Phi_{\CalP}}

\newcommand{\TP}{\mathsf{T}_\CalP}
\newcommand{\Lfp}{\mathsf{lfp}\,}
\newcommand{\CalAPTwo}{\CalA_{2,\CalP}}
\newcommand{\CalETwo}{\CalE}
\newcommand{\two}{two-valued~} 
\newcommand{\TPmodels}{\models}
\newcommand{\Lmtwo}{\mathsf{lm}_2}


\newcommand{\CalAP}{\CalA_\CalP}
\newcommand{\CalOP}{\CalO_\CalP}
\newcommand{\CalA}{{\cal A}}
\newcommand{\CalB}{{\cal B}}
\newcommand{\CalC}{{\cal C}}
\newcommand{\CalD}{{\cal D}}
\newcommand{\CalE}{{\cal E}}
\newcommand{\CalG}{{\cal G}}
\newcommand{\CalH}{{\cal H}}
\newcommand{\CalF}{{\cal F}}
\newcommand{\CalI}{{\cal I}}
\newcommand{\CalIC}{{\cal IC}}
\newcommand{\CalK}{{\cal K}}
\newcommand{\CalL}{{\cal L}}
\newcommand{\CalM}{{\cal M}}
\newcommand{\CalN}{{\cal N}}
\newcommand{\Call}{\tt Call}
\newcommand{\CalO}{{\cal O}}
\newcommand{\CalP}{{\cal P}}
\newcommand{\CalR}{{\cal R}}
\newcommand{\CalS}{{\cal S}}
\newcommand{\CalT}{{\cal T}}
\newcommand{\CalU}{{\cal U}}
\newcommand{\CalV}{{\cal V}}
\newcommand{\CalW}{{\cal W}}
\newcommand{\CalX}{{\cal X}}
\newcommand{\CalZ}{{\cal Z}}


\newcommand{\IC}{\textsf{IC}}

\newcommand{\fly}{\mathit{fly}}
\newcommand{\bird}{\mathit{bird}}
\newcommand{\abnormal}{\mathit{abnormal}}
\newcommand{\irregular}{\mathit{irregular}}
\newcommand{\regular}{\mathit{regular}}
\newcommand{\cycle}{\mathit{cycle}}
\newcommand{\negodd}{\mathit{neg}\text{-}\mathit{odd}}
\newcommand{\negcycle}{\mathit{neg}\text{-}\mathit{cycle}}
\newcommand{\nA}{\mathsf{n}\_A}
\newcommand{\modif}{\mathsf{mod}}
\newcommand{\invmodif}{\mathsf{invmod}}


\newcommand{\cxt}{\mathsf{ctxt}}
\newcommand{\inspect}{\mathsf{inspect}} 
\newcommand{\inspectneg}{\mathsf{inspect_{not}}} 
\newcommand{\insp}{\mathsf{insp}}


\newcommand{\pre}{\mathit{pre}}

\newcommand{\add}{\mathit{add}}
\newcommand{\addictive}{\mathit{addictive}}
\newcommand{\nadd}{\mathit{addictive^\prime}}
\newcommand{\inex}{\mathit{inex}}
\newcommand{\inexpensive}{\mathit{inex}}
\newcommand{\ninex}{\mathit{inex^\prime}}
\newcommand{\cig}{\mathit{cig}}
\newcommand{\cigarette}{\mathit{cig}}
\newcommand{\ab}{\mathit{ab}}
% \newcommand{\abnadd}{\mathit{ab}}
% \newcommand{\abinex}{\mathit{ab}}
\newcommand{\abnmil}{\mathit{ab}}
\newcommand{\abnnut}{\mathit{ab}}
\newcommand{\abnadd}{\mathit{ab_{\add^\prime}}}
\newcommand{\abinex}{\mathit{ab_{\inex}}}


\newcommand{\mil}{\mathit{mil}}
\newcommand{\millionaire}{\mathit{mil}}
\newcommand{\nmillionaire}{\millionaire^\prime}
\newcommand{\rich}{\mathit{rich}}
% \newcommand{\abnmil}{\mathit{ab_{\mil^\prime}}}
\newcommand{\work}{\mathit{work}}
\newcommand{\hardworker}{\mathit{hard\_worker}}

\newcommand{\dog}{\mathit{dog}}
\newcommand{\poldog}{\mathit{police\_dog}}
\newcommand{\vicious}{\mathit{vicious}}
\newcommand{\abdog}{\mathit{ab_{dog^\prime}}}
\newcommand{\hight}{\mathit{highly\_trained}}


% \usepackage{scalerel}

\newcommand{\vit}{\mathit{vit}}

\newcommand{\vitamin}{\mathit{vitamin}}
\newcommand{\nutritional}{\mathit{nutritional}}
\newcommand{\nnutritional}{\mathit{nutritional^\prime}}
% \newcommand{\abnnut}{\mathit{ab_{nut^\prime}}}
\newcommand{\abnut}{\mathit{ab_{nut}}}

\newcommand{\abhigh}{\mathit{ab_{trained}}}
\newcommand{\abrich}{\mathit{ab_{rich}}}

\newcommand{\storm}{\mathit{storm}}
\newcommand{\lightning}{\mathit{lightning}}
\newcommand{\tempest}{\mathit{tempest}}
\newcommand{\ffire}{\mathit{ffire}}
\newcommand{\barbecue}{\mathit{barbecue}}
\newcommand{\dry}{\mathit{dry\_leaves}}
\newcommand{\drys}{\mathit{dry}}
\newcommand{\rained}{\mathit{rained}}
\newcommand{\smoke}{\mathit{smoke}}
\newcommand{\fire}{\mathit{fire}}
\newcommand{\ffighters}{\mathit{ffighters}}
\newcommand{\sirens}{\mathit{sirens}}
\newcommand{\sir}{\mathit{sir}}



\newcommand{\Make}{\mathsf{make}} 

\newcommand{\lights}{\mathit{light}}
\newcommand{\fofi}{\mathit{forest\_fire}}


\newcommand{\died}{\mathit{died}}
\newcommand{\thi}{\mathit{died\_of\_thi}}
\newcommand{\poisshort}{\mathit{pois}}
\newcommand{\pois}{\mathit{died\_of\_pois}}
\newcommand{\hole}{\mathit{hole}}
\newcommand{\wat}{\mathit{water\_pois}}


\newcommand{\ja}{\mathit{jb}}
\newcommand{\born}{\mathit{born}}
\newcommand{\traitor}{\mathit{trait}}
\newcommand{\trait}{\mathit{trait}}
\newcommand{\communist}{\mathit{com}}
\newcommand{\com}{\mathit{com}}
\newcommand{\uk}{\mathit{uk}}
\newcommand{\russia}{\mathit{rus}}

\newcommand{\AUXATOMS}{\mathsf{AUXATOMS}}

\newcommand{\cons}{\mathsf{c}}
\newcommand{\consO}{\cons_1}
\newcommand{\consT}{\cons_2}
\newcommand{\atoms}{\mathsf{atoms}}
\newcommand{\Ab}{ab}
\newcommand{\Body}{\mbox{\em Body}}

\newcommand{\defname}[1]{\textit{#1}}
\newcommand{\name}[1]{\textit{#1}}

\newcommand{\pose}{e}
\newcommand{\posl}{l}
\newcommand{\nege}{\lnot e}
\newcommand{\negl}{\lnot l}


\newcommand{\Ground}{\mathsf{g}}

\newcommand{\LSEM}{\textsf{\L}-semantics\xspace}
\newcommand{\SSEM}{\textsf{S}-semantics\xspace}
\newcommand{\SvLSEM}{\textsf{SvL}-semantics\xspace}
\newcommand{\SSS}{$\mathsf{S_3}$\xspace}


\newcommand{\WComp}{\mathsf{wc}\,}


\newcommand{\SThree}{{\mbox{\tiny S}}}

\newcommand{\WFM}{\mathsf{wfm}_{\SThree}\,}

%% ch: \mathsf{l} sieht leider wie ein Strich oder I aus
%% warum eigentlich nicht ohne mathsf, es ist ja keine konkrete
%% funktion oder operator\
% \renewcommand{\Level}{\mathsf{level}}
\newcommand{\Level}{l}
\newcommand{\LevelI}{l_I}


\newcommand{\auxiliary}{\xspace{auxiliary}}

\newcommand{\eqdef}{=}
\newcommand{\revimp}{\leftarrow}
\newcommand{\equi}{\leftrightarrow}

\newcommand{\posbody}{\mathsf{pos}}
\newcommand{\negbody}{\mathsf{neg}}

% \newcommand{\ATOMS}{\mathcal{A}} % global set of all atoms
\newcommand{\ATOMS}{\mathsf{ATOMS}} % global set of all atoms

% \renewcommand{\top}{T}

\newcommand{\test}{\boldmath{test}}
\newcommand{\true}{\mathbf{\top}}
\newcommand{\false}{\mathsf{\bot}}
\newcommand{\udf}{\mathsf{U}}

\newcommand{\Luka}{{\mbox{\tiny\L}}}

\newcommand{\LeastMod}{\mathsf{lm}}
\newcommand{\Lmwcl}{\LeastMod_{\Luka}\mathsf{wc}\,}
\newcommand{\ModelsLMWC}{\models_{\Luka}^{\LeastMod\mathsf{wc}\,}}
\newcommand{\ModelsWCS}{\models_{wcs}}
\newcommand{\ModelsL}{\models_{\mbox{\tiny\L}}}

\newcommand{\NATURAL}{\mathbb{N}}

\newcommand{\modelss}{\models_s^{\LeastMod\mathsf{wc}\,}}
\newcommand{\modelsc}{\models_c^{\LeastMod\mathsf{wc}\,}}

\newcommand{\MA}{\textsf{A}}
\newcommand{\MI}{\textsf{I}}
\newcommand{\ME}{\textsf{E}}
\newcommand{\MO}{\textsf{O}}
\newcommand{\NVC}{\textsf{NVC}}

\newcommand{\Def}{\mathsf{def}}

\newcommand{\WC}{\mathit{wc}}

\newcommand{\WCS}{\textsc{WCS}\xspace}

\newcommand{\True}{\textit{true}\xspace}
\newcommand{\False}{\textit{false}\xspace}


\newcommand{\basicc}{\textsf{basic}}
\newcommand{\contrac}{\textsf{contraposition}}
\newcommand{\abdc}{\textsf{abduction}}
\newcommand{\genc}{\textsf{generalization}}
% \newcommand{\basiccell}{\textsf{basic}}


\newcommand{\betweentinyandsmall}{\fontsize{6}{8}\selectfont}
\newcommand{\Pie}{{\CalP_{\tiny\mbox{\MI\ME 2}}}}
\newcommand{\Piecontra}{{\CalP^{\tiny \contra}_{\tiny\mbox{\MI\ME 2}}}}
\newcommand{\Piegen}{{\CalP^{\tiny \negFail}_{\tiny\mbox{\MI\ME 2}}}}
\newcommand{\conditionals}{{\small\textsf{conditional}}}
\newcommand{\licenses}{{\small\textsf{license}}}
\newcommand{\import}{{\small\textsf{import}}}
\newcommand{\transformation}{{\small\textsf{transformation}}}
\newcommand{\dnegation}{{\small\textsf{noDoubleNeg}}}
\newcommand{\unknownGen}{{\small\textsf{unknownGen}}}
\newcommand{\abduction}{{\small\textsf{searchAlt}}}
\newcommand{\converse}{{\small\textsf{converse}}}
\newcommand{\negFailure}{{\small\textsf{generalization}}}
\newcommand{\contraposition}{{\small\textsf{contra}}}
\newcommand{\contra}{{\small\textsf{contra}}}
\newcommand{\negFail}{{\small\textsf{gen}}}
\newcommand{\basic}{{\small\textsf{basic}}}


\newcommand{\ex}{\textsf{ex}}
\newcommand{\imp}{\textsf{}}
\newcommand{\Answers}{\textsf{Answers}}
\newcommand{\Answer}{\textsf{FinalAnswer}}


% \newcommand{\commentem}[1]{
% \medskip \noindent
% {\sc COMMENT (EMMA):
% #1}
% }

\parskip 0.2in
\linespread{1.25}


\thispagestyle{empty}

\title{Cognitive Principles and Individual Differences \\ in Human Syllogistic Reasoning}
\author{
Emmanuelle-Anna Dietz Saldanha and Richard M{\"o}rbitz\thanks{The authors are mentioned in alphabetical order.} \\
\normalsize International Center for Computational Logic, TU
  Dresden, Germany
}  
% \institute{}
\date{}

\begin{document}
\maketitle

\section{Introduction}
% 
% 
% Technology occupies a large part of every day life in our society.
% Therefore, an effortless interaction between humans and technical devices
% is of crucial interest for everyone. As it is not likely that humans will make the effort to understand their devices but instead they will place high demands on them, their devices should be able to adapt to the human needs in the best possible way. For instance, a system should be able to recognize small changes and additionally identify which environmental parameters are relevant for the current communication with its owner. This in turn should allow the device to adapt flexibly and independently to the environment it is currently situated in. In order to develop such a system, it is necessary to understand how humans think and to recognize their behavioral patterns. 

Even though the main logical formalism of Classical Logic (CL) has successfully been applied to our scientific reasoning,
various experimental studies have shown that human reasoning systematically deviates 
from conclusions that are valid under CL (cf.~\cite{wason:68,byrne:89}).
% Originally, one of the objectives in the field of AI and logic programming was to formalize human reasoning and commonsense reasoning~\cite{McCarthy59,McCarthy98}. 
% However, the proposed approaches have been strongly critized~(cf.~\cite{mcdermott:87}).
% On the other hand, cognitive scientists have extensively investigated and acquired a great expertise
% in human thinking. Recently, several attempts have been made to bridge the gap among these disciplines. 
% % as they were mostly theoretical and did not apply to real case studies.
% However, only recently an attempt was made 
% this expertise was not consulted, i.e. there was no communication among the relevant fields. 
Cognitive theories that try to explain human reasoning need to be evaluated by their cognitive adequacy~\cite{strube:1992} and their performance should be compared to other theories.
Conceptual cognitive adequacy describes to what extent a system corresponds to human conceptual knowledge, whereas inferential cognitive adequacy indicates whether the reasoning process of a system is structured similarly to the way humans reason~\cite{knauff:rauh:renz:1997}. 
In order to determine the required factors that are necessary for the evaluation of an adequate cognitive theory, we take~the suggestion by Ragni and Stolzenburg~\cite{ragni:stolzenburg:2015} as a starting point:
We aim for a theory in which (i) all processes are comprehensible, (ii) the suggested underlying represent general cognitive processes, and (iii) the predictions of our theory are evaluated with respect to a wide range of human reasoning tasks. 

% Yet, the cognitive adequacy of a theory can only be measured, if that theory is evaluable, i.e.\ if the theory can uniformally account for episodes of human reasoning systematically, according to its underlying assumptions. 
% Only then, the cognitve adequacy of that theory can be compared to others.
% 
% Formal methods can offer the required tools to implement such cognitive theories. This in turn, can allow a new way of communication among the fields, by avoiding ambiguous definitions, showing their formal correspondences and finally providing appropriate evaluation methods with respect to their cognitive adequacy. 
% A major drawback of 
The results of the meta-study on syllogistic reasoning by Khemlani and Johnson-Laird~\cite{khemlani:2012} show that the answers given by humans do not only systematically differ from the valid conclusions under CL, but from any other proposed cognitive theory (cf.~\cite{BJ89,johnsonlaird:1983,Rips1994,Polk1995,chater:oaksford:1999}).
% Possibly one of the reasons could be that none of the suggested cognitive theories so far has accounted for differences among individuals, i.e.\ they only considered the data summed up with respect to the individual answer possibilities. 
% As recent investigations on psychological experiments show (cf.~\cite{khemlani:2016, ragni:2017}), it seems that different humans represent and reason about conditional information differently.
% This implies that \textit{the} human reasoner does not exist, instead there 
% seem to exist various (human) reasoning clusters.


% The approach suggested in~\cite{stolzenburg:luederitz:2017}, is based on venn diagrams, where moods are understood as conditional probabilities, and implemented 
% in constraint logic programming. Yet, another logic programming approach, the Weak Completion Semantics, is suggested in~\cite{cogsci:2017}.
Syllogisms originate from Aristotle~\cite{aristotle} and
they are interesting because on one hand they are easy to understand and on the other hand
they are complex enough to require actual reasoning. 
A syllogism consists of two
premises and one conclusion,
each of which is a statement about properties over terms which can have one of the classical quantifiess in Table~\ref{table:moods} (called \emph{moods}). The pair of syllogistic premises can have four different orders as shown in Table~\ref{t:2} (called \emph{figures}). Based on these moods and figures, 64 different pairs of syllogistic premises can be constructed.
A conclusion then consists of an assertion about two properties over terms where each of these properties occurs in exactly one of the premises, and there is a third property connecting both premises which does not occur in the conclusion.
\begin{table}[t]
    \begin{minipage}{0.6\textwidth}
        \centering
	\[	
\begin{array}{@{\hspace{0.0cm}}l@{\hspace{0.1cm}}G@{\hspace{0.1cm}}l@{\hspace{0.3cm}}c}\toprule
	\mbox{Mood} & \mbox{natural language} & \mbox{First-order Logic} &
\mbox{Short} \\\midrule
	\mbox{affirmative universal} & \mbox{\em all a are b} & \forall
	X (a(X) \rightarrow b(X)) & {\MA ab}\\
	\mbox{affirmative existential} & \mbox{\em some a are b} &
	\exists X (a(X) \wedge b(X)) & {\MI ab}\\
	\mbox{negative universal} & \mbox{\em no a are b} &
	\forall X (a(X) \rightarrow \neg b(X)) & {\ME ab}\\
	\mbox{negative existential} & \mbox{\em some a are not b} &
	\exists X (a(X) \wedge \neg b(X)) & {\MO ab} \\
	\bottomrule
	\end{array}
	\]
	\caption{The four moods and their
formalization\label{table:moods}.}
	\end{minipage}
    \begin{minipage}{0.45\textwidth}
        \centering
%  	\small
	\begin{center}
	\[
		\begin{array}{@{\hspace{0cm}}c@{\hspace{0.2cm}}c@{\hspace{0.1cm}}c@{\hspace{0.1cm}}c@{\hspace{0.1cm}}c}
		\toprule
		 \mbox{Figure} & \mbox{Premise 1} & \mbox{Premise 2} \\
		\midrule
		\mbox{1} &  a-b & b-c \\
		\mbox{2} & b-a &  c-b \\
		\mbox{3} & a-b & c-b \\
		\mbox{4} & b-a & b-c \\
		\bottomrule
		\end{array}
		\]
	\end{center}
	\caption{The four figures.\label{t:2}}
	\end{minipage}
\end{table}
For instance, consider the following pair of syllogistic premises: 
\begin{align}
 \emph{Some bakers are artists. }\quad\quad  \quad\quad & \emph{No chemists are bakers.} \tag{IE2} 
\end{align}
Given these
two premises, which conclusion on the relation between \emph{artists} and \emph{chemists} necessarily follows?
According to Khemlani and Johnson-Laird~\cite{khemlani:2012}, the majority of participants in experimental studies
concluded \emph{No chemists are artists} (29\;\%) and \emph{No valid conclusion} (27\;\%).
The valid conclusion under CL, \emph{Some artists are not chemists},
was concluded by only 16\,\%, which is exactly the threshold that the answer had not been chosen randomly~\cite{khemlani:2012}.
% \footnote{
% See
%  Given that there are nine different possible conclusions, the chance that a conclusion has been chosen randomly is 1/9 = 11.1 \%. 
%  A binomial test shows that if a conclusion is drawn in
% more than 16\% of the cases by the participants it is unlikely that is has been
% chosen by just random guesses~\cite{khemlani:2012}. }
% Recently, a new cognitive theory has been proposed for syllogistic reasoning~\cite{cogsci:2017}.
% Surprisingly this approach is competitive with the other cognitive theories discussed in~\cite{khemlani:2012}.
% \cite{cogsci:2017} deals with the first step of modeling human reasoning towards an adequate representation, for which it identifies seven cognitive principles, i.e.\ assumptions that humans make while reasoning, motivated from the literature and own observations. For each of these principles a formal representation
% in logic program clauses is provided. 
% For the second step,  reasoning towards the least models of the weak completion of these programs under the three-valued
% {\L}ukasiewicz Logic, is suggested.
% One reason is because often only {aggregated data} of the experiments are accessible.
% i.e.\ the data summed up with respect to the individual answer possibilities.
% The experimental set-up was made such that participants would only choose one possible conclusion, which implies that 
% these answers excluded each other. In particular, it can be assumed that participants were at least minimally consistent in the sense that
% i.e.\ 
% The majority's conclusions exclude each other, as the test was made such that
Usually in these experiments, participants are supposed to give only one answer, thus all three (significant) answers
were given by different participants. 
% The ones who answered \emph{No valid conclusion}
% were not the same ones who answered \emph{No chemists are artists} or \emph{Some artists are not chemists}, and vice versa.\footnote{}

A major drawback of the suggested cognitive theories so far is that only the participants'
aggregated data was considered: These theories did not account for differences
among individuals, i.e.\ they only considered the data summed up with respect to the individual answer possibilities. In contrast, non-aggregated data records the individual participants' response patterns, i.e.\ the complete set of responses that are given by each participant.


As the above example and recent investigations on psychological experiments show (cf.~\cite{khemlani:2016, ragni:2017}), it seems that different humans represent and reason about conditionals differently.
This implies that \textit{the} human reasoner does not exist, instead there 
seem to exist various (human) reasoning clusters. The identification of these differences and the specification of these clusters is central 
for the development of an adequate cognitive theory. Only recently approaches have been proposed that account for individual differences (or clusters) in syllogistic reasoning (cf.~\cite{khemlani:2016,declare:2017}). 


%  The results show that some but not all of the proposed clusters are well reflected in the individual reasoning patterns.
% More interestingly are the following two observations:
% The dataset in~\cite{khemlani:2016} provides an additional interesting information: 
% Interestingly, when comparing the changes of results of each individual participant
% from the first to the second week, it seems that their answers in the second week fits better the classical logic derivations. 
% We argue that these reasoners can be characterized by a set of cognitive principles, motivated from studies made in Cognitive Science.
% First attempts to propose a model that distinguishes between human reasoners have been proposed in~\cite{khemlani:2016}. 
% The authors identify three different clusters of reasoners labeled as intuitive, intermediate and deliberative. They propose an approaches based on the MReasoner and suggest to   ...
% We evaluate the non-aggregated data provided by~\cite{khemlani:2016} and taking the clusters suggested in~\cite{declare:2017} as starting point. The non-aggregated data consists of two dataset for 20 participants, who carried out the same syllogistic reasoning task twice with a interval of one week.
%  The evaluation shows that some but not all of the proposed clusters in~\cite{declare:2017} are well reflected in the individual reasoning patterns.
% Interestingly, when comparing the changes of results of each individual participant
% from the first to the second week, it seems that their answers in the second week fits better the classical logic derivations. 
% We formulate the hypothesis that the individual clusters do not represent different human reasoners, but different stages of reasoning. Depending on the persistence and the
% previous training of the individual participant, ...

Here, we focus on a qualitative assessment of the clustering approach proposed in~\cite{declare:2017},
in the sense that we investigate the individual participants' conclusions with respect to the non-aggregated dataset provided by Johnson-Laird and Khemlani~\cite{khemlani:2016}.
In Section~\ref{sect:wcsclustering} and~\ref{sect:representing}, we introduce the underlying computational logic approach and the formal representation of cognitive principles within this approach. Section~\ref{sect:aggregateddata} illustrates an example and briefly discusses the performance of the approach presented in~\cite{declare:2017} on non-aggregated data.
Section~\ref{sect:nonag} provides a more adequate characterization of the individuals' reasoning patterns.

\section{Clustering by Principles under WCS} \label{sect:wcsclustering}

Clustering by principles under WCS was first presented in~\cite{declare:2017};
the authors conjectured that reasoning clusters can be expressed through cognitive principles and heuristic strategies. Cognitive principles are assumptions (not necessarily valid under CL) made by humans while reasoning, 
whereas heuristic strategies describe the methods that are applied when
conclusions are not derived by reasoning but instead based on structural patterns or {biases}.
This approach is an extension of the computational logic approach for human syllogistic reasoning presented in~\cite{cogsci:2017}: Three reasoning clusters and two heuristic strategies were specified, and achieved a match of 92\;\%, which was the highest result compared to the presented approaches by Khemlani an Johnson-Laird~\cite{khemlani:2012}.
% The system developed in~\cite{declare:2017}, was extended and participated at the Syllogism Challenge 2017.\footnote{\url{https://www.cc.uni-freiburg.de/modelingchallenge/challenge-2017}} 
Figure~\ref{overview:system:syllchal} illustrates the components of the system:
First, groups of reasoners were specified based on a set of underlying cognitive principles and heuristic strategies.
After that, this set was translated into a logic program (Section~\ref{sect:basic}-\ref{sect:advanced}), representing the group of reasoners.
Next, the respective models were computed (Section~\ref{sect:wcs}-\ref{sect:lfp}).
After that, the entailed relations between $a$ and $c$ were extracted (Section~\ref{sect:entail}). 
 The final answer distribution was specified according to the number of reasoning clusters 
that entailed the conclusion.
Finally, an illustrative representation of these clusters with respect to their distribution was done by means of multinomial process trees~\cite{Riefer1988}.

\begin{figure}\small
\begin{tikzpicture} 
\draw (0,8.5) rectangle (13.2,9.5);
\node (A1) at (2,8.4) {};

\node[anchor=west]  at (0,9.2) {\textbf{Cognitive Principles} presuppositions that humans make while reasoning};
\node[anchor=west]  at (0,8.7) {\textbf{Heuristic Strategies} reasoning based on structure in patterns or biases};

\node (A2) at (2,7.9) {};
\draw[thick, ->] (A1) to[m] node[right] {\tiny \gcolor{Description}} (A2);

\draw (0,7.3) rectangle (13.2,7.8);
\node[anchor=west] at (0,7.5) {\textbf{Reasoning Clusters} groups of reasoners};
\node (A3) at (2,7.3) {};

\node (A4) at (2,6.8) {};
\draw[thick, ->] (A3) to[m] node[right] {\tiny \gcolor{Translation}} (A4);
% 
\draw (0,6.3) rectangle (13.2,6.8);
\node[anchor=west]  at (0,6.5) {\textbf{Program $\CalP$} cognitive principles represented
by logic program clauses};
\node (A5) at (2,6.3) {};

\node (A6) at (2,5.8) {};
\draw[thick, ->] (A5) to[m] node[right] {\tiny \gcolor{Computation}} (A6);

\draw (0,5.3) rectangle (13.2,5.8);
\node[anchor=west]  at (0,5.5) {\textbf{Model $\CalM$} least fixed point of the semantic $\Phi_\CalP$ operator};
\node (A7) at (2,5.3) {};

\node (A8) at (2,4.8) {};
\draw[thick, ->] (A7) to[m] node[right] {\tiny \gcolor{Entailment}} (A8);

\node[anchor=west]  at (0,4.5) {\textbf{Conclusions} all artists are chemists (\MA ac), some artists are chemists (\MI ac), .... };
\draw (0,4.3) rectangle (13.2,4.8);
\node (A9) at (2,4.3) {};

\node (A10) at (2,3.8) {};
\draw[thick, ->] (A9) to[m] node[right] {\tiny \gcolor{Specification}} (A10);

\draw (0,3.2) rectangle (13.2,3.8);
\node[anchor=west]  at (0,3.5) {\textbf{Distribution} based on number of clusters that entail the conclusion};

\draw (13.4,5.3) rectangle (15.6,7.3);
\node (A11) at (14.5,6.7) {{\small Multinomial}};
\node at (14.5,6.3) {\small Process};
\node at (14.5,5.8) {\small Trees};
\draw[thick, ->] (13.2,9) -| node {}  (14.5,7.3);
\draw[thick, ->] (14.5,5.3) |- node {}  (13.2,3.55);


\end{tikzpicture}
\caption{The system that participated at the Syllogism Challenge 2017\label{overview:system:syllchal}.}

\end{figure}  

The suggested computational logic approach for the formal representation and reasoning of these clusters,
is the Weak Completion Semantics (WCS)~\cite{hk:2009a,hk:2009b}
which has shown to model a wide range of human reasoning tasks~\cite{btg:2015:h}. 
% WCS is the non-monotonic computational logic approach to consider
% weakly completed logic programs, to compute their least models
% under the three-valued {\L}ukasiewicz logic~\cite{lukasiewicz:20}, and to reason with respect to these models or search for explanations by means of (skeptical) abduction~\cite{kakas:etal:93}.

\subsection{Contextual Logic Programs} \label{sect:wcs}

We introduce an extension of WCS first presented in~\cite{dietz:hoelldobler:pereira:2017}. 
\textit{Contextual (logic) program clauses}
are expressions of the forms $A  \leftarrow  L_1 \land \ldots \land L_m \land
\cxt(L_{m+1}) \land \ldots \land \cxt(L_{m+p})$ (called
\textit{rules}), $A \leftarrow \top$ (called \textit{facts}), $A \leftarrow \bot$ (called
\textit{negative assumptions})\footnote{Under WCS, the negative
assumption will become $A \leftrightarrow \bot$, which, however, can be overwritten by other rules and facts (\textit{defeating} the assumption).} and $A \leftarrow \udf$ (called
\textit{unknown assumptions}), where
$A$ is an atom and~$L_i$ with~$1 \leq i \leq m+p$, is a literal.
$A$ is called \textit{head}
and $L_1 \land \ldots \land L_m \land
\cxt(L_{m+1}) \land \ldots \land \cxt(L_{m+p})$ as well as~$\top, \bot$
and $\udf$,
standing for \textit{true}, \textit{false} and \textit{unknown} respectively, are
called \textit{body} of the corresponding clauses.  A
\textit{contextual program}, denoted by $\mathcal P$,
is a finite set of contextual program clauses.
We restrict terms to be either constants or variables only, hence we consider so-called \defname{data logic programs}.
If a clause contains variables, then they are implicitly universally quantified within the scope of the entire clause. 
% For instance, the clause $p(X) \leftarrow q(X)$ represents the formula $(\forall X) (p(X) \leftarrow q(X))$. Therefore, a clause never contains free variables.
A \defname{ground clause} is a clause that does not contain variables.
$\Ground \CalP$ denotes the set of all ground clauses of $\CalP$.
An atom $A$ is \textit{defined in $\Ground \CalP$}  iff $\Ground \CalP$ contains a clause with head $A$.
The \textit{definition of $A$ in $\Ground \CalP$} is defined as $\Def(A,\CalP)
= \{ A \leftarrow \Body \mid A \leftarrow \Body \mbox{ is a clause
occurring in } \Ground \CalP\}$.
$A$ is \textit{undefined in $\Ground \CalP$} iff $A$ is not defined in $\Ground \CalP$.
The set of all atoms that are undefined in~$\Ground \CalP$ is denoted by~$\ud(\CalP)$.
Consider the following transformation for a given $\CalP$:
1.\ For each atom~$A$ defined in $\Ground \CalP$, replace
  all clauses of the form $A \leftarrow \Body_1, \ldots,$ $A \leftarrow
  \Body_m$ occurring in $\Ground \CalP$ by $ A \leftarrow \Body_1 \vee
  \ldots \vee \Body_m$. 2.\ Replace all occurrences of $\leftarrow$ by $\leftrightarrow$.
The obtained ground set of equivalences is called the \textit{weak completion}
of~$\CalP$ or~$\WC\CalP$.
% $\neg A$ is \textit{assumed in $\Ground \CalP$} iff $\Ground \CalP$
% contains a negative assumption with head $A$, $\Ground \CalP$ does not
% contain an unknown assumption with head $A$, and $\Def(A,\CalP) = \emptyset$.
We omit the word \textit{contextual} when we refer to programs, if not stated otherwise.

\subsection{Three-Valued {\L}ukasiewicz Logic Extended with $\cxt$} \label{sub:3valuedL}

We consider the three-valued {\L}ukasiewicz logic~\cite{lukasiewicz:20}
extended with the $\cxt$ connective, for which the
corresponding truth values are $\top$, $\bot$ and $\udf$, meaning \textit{true}, \textit{false}
and \textit{unknown}, respectively.
% As it seems,  $\cxt$ corresponds to {\L}ukasiewicz necessity operator~\cite{}, where the difference is, that $\cxt$ is restricted to be applicable to literals only.
A \defname{three-valued interpretation}~$I$ is a mapping from the set of
logical formulas to the set
of truth values~$\{ \true, \bot, \udf\}$, represented as a pair~$I~=~\langle I^\true,I^\bot \rangle$ of two disjoint sets of atoms:  $I^\true =  \{A \mid A \mbox{ is mapped to } \true \mbox{ under } I\}$ and $I^\bot = \{A \mid A \mbox{ is mapped to } \bot \mbox{ under } I \}$.
Atoms which do not occur in~$I^\true \cup I^\bot$ are mapped to~$\udf$.
The truth value of a given formula
under $I$ is determined according to the truth tables in
Table~\ref{tab:3vldluka}.
\footnote{Thanks to the
 anonymous reviewer for the observation, that the $\cxt$ operator seems to be a special case of {\L}ukasiwicz's necessity operator~\cite[p. 20]{malinowski:1993}, restricted to literals.
 }
 
$I(F) = \top$ denotes that a formula $F$ is mapped to true under $I$.
A \defname{three-valued model} $\CalM$ of~$\CalP$ is a three-valued interpretation such that
$\CalM(A \leftarrow \Body) = \true$ for each $A \leftarrow \Body \in \Ground\CalP$.
Let $I = \langle I^\top, I^\bot \rangle$ and $J = \langle J^\top,
J^\bot \rangle$ be two interpretations.
$I \subseteq J$ iff $I^\top \subseteq J^\top$ and $I^\bot \subseteq J^\bot$. $I$ is the \textit{least model} of $\CalP$
iff for any other model~$J$ of $\CalP$ it holds that $I \subseteq J$.

\begin{table}[t]
\[\small
\begin{array}[c]{@{\hspace{0mm}}c|c}
F & \neg F\\ \midrule
\top & \bot \\
\bot & \top \\
\udf & \udf
\end{array}
\;\
\begin{array}{c|lll}
 \wedge & \top & \udf & \bot \\
\midrule
\top & \top & \udf & \bot \\
\udf & \udf & \udf & \bot \\
\bot & \bot & \bot & \bot
 \end{array}
\;\
\begin{array}{c|lll}
 \vee & \top & \udf & \bot \\
\midrule
\top & \top & \top & \top \\
\udf & \top & \udf & \udf \\
\bot & \top & \udf & \bot
 \end{array}
\;\
\begin{array}{c|lll}
\leftarrow & \top & \udf & \bot \\
\midrule
\top & \top & \top & \top \\
\udf & \udf & \top & \top \\
\bot & \bot & \udf & \top
 \end{array}
\;\
\begin{array}{c|ccc}
  \leftrightarrow  & \top & \udf & \bot \\
  \midrule
\top & \top & \udf & \bot \\
\udf & \udf & \top & \udf\\
\bot & \bot & \udf & \top
 \end{array}
\;\
\begin{array}{c|c}
 L & \cxt(L)\\
\midrule
\top & \top \\
\bot & \bot  \\
\udf & \bot
\end{array}
\]
\caption{The truth tables for the connectives under the three-valued {\L}ukasiewicz logic and for $\cxt$. $L$ is a literal and $\top$, $\bot$, and $\udf$ denote \textit{true}, \textit{false},
and \textit{unknown}, respectively.\label{tab:3vldluka}}
\end{table}

A set of \textit{integrity constraints} $\CalIC$ consists of clauses of
the form $\udf \leftarrow \Body$, where $\Body$ is a conjunction of
literals. An interpretation maps
an integrity constraint to $\top$ iff $\Body$ is either mapped to
$\udf$ or $\bot$.  
Given an interpretation~$I$
and a set of integrity constraints~$\CalIC$,~$I$ \textit{satisfies}~$\CalIC$
iff all clauses in~$\CalIC$ are true under~$I$.


\subsection{Reasoning with Respect to the Least Fixed Point of $\Svlp$}\label{sect:lfp}


Consider the following semantic operator introduced by Stenning and van Lambalgen~\cite{stenning:vanlambalgen:2008}: Let $I = \langle I^\top,I^\bot
\rangle$ be an interpretation. $\Phi_\CalP (I) =
\langle J^\top,J^\bot\rangle$, where
\[
\begin{array}{lcl}
J^\top & = & \{ A \mid A \leftarrow \Body \in \Def(A,\CalP)
\mbox{ and
} \Body \mbox{ is \True under } \langle I^\top,I^\bot \rangle \} \\
J^\bot & = & \{ A \mid \Def(A,\CalP) \neq \emptyset \mbox{ and }\\
&& \rule{7.5mm}{0mm}\Body \mbox{ is \False under } \langle
I^\top,I^\bot \rangle \mbox{ for all }  A \leftarrow \Body \in \Def(A,\CalP)\}.
\end{array}\Shrink
\]
The weak completion of non-contextual programs\footnote{Non-contextual programs refer programs that do not contain rules in which $\cxt$ appears in the body of any rule, i.e.\ $p = 0$.} 
always has a least model under  {\L}ukasiewicz logic,
which can be computed by the least fixed point of~$\Svlp$~\cite{hk:2009a}. The correspondence to other well-known logic programming approaches, in particular 
to the well-founded semantics has been shown in~\cite{dietz:hoelldobler:wernhard:2014}.
 
The weak completion of contextual
programs might have more than one minimal model, and it is not guaranteed that 
the least fixed point of $\Svlp$ always exists. However,
for programs without cycles, there always exists a unique supported model,
which corresponds to the least fixed point of~$\Svlp$~\cite{dietz:hoelldobler:pereira:2017,dietz:hoelldobler:philipp:2017}. This model is \textit{supported} in the sense that 
it will always be computed by $\Svlp$, independent from the interpretation we start to iterate $\Svlp$ with. Let us denote this model as $\CalM_\CalP$:
% As the fixed point of $\Svlp$ is unique and only computes the supported minimal model of $\WComp\CalP$, 
% we define 
$\CalP \ModelsWCS F$ iff $\CalM_\CalP(F) = \top$.
For all programs considered in the sequel, $\CalM_\CalP$ exists.

% % \subsection{Backward Reasoning with respect to Skeptical Abduction} \label{sect:abduction}
% % 
% % An \textit{abductive framework} $\langle \CalP, \CalA, \CalIC, \ModelsWCS \rangle$
% % consists of a program $\CalP$, a set~$\CalA$ of abducibles, a set $\CalIC$ of
% % integrity constraints, and the entailment relation $\ModelsWCS$.  The set of
% % abducibles is~$\CalA =\{ A \leftarrow \top \mid A \in \ud(\CalP) \} \ \cup \ \{ A
% % \leftarrow \bot \mid A \in \ud(\CalP) \mbox{ and $\neg A$ is not assumed in
% % $\Ground \CalP$}\}$.
% % Let $\langle \CalP, \CalA, \CalIC, \ModelsWCS \rangle$ be
% % an abductive framework and the observation~$\CalO$ a set of literals.
% % $\CalO$ is \emph{explainable} in $\langle \CalP, \CalA, \CalIC, \ModelsWCS
% % \rangle$ iff there exists an $\CalE \subseteq \CalA$, such that
% % $\mathcal{P} \cup \mathcal{E} \models L$ for all $L \in \mathcal{O}$ and
% % $\mathcal{P} \cup \mathcal{E}$ satisfies $\CalIC$.  $\mathcal{E}$ is then
% % called \emph{explanation} for $\CalO$ given $\CalP$ and $\CalIC$.  We restrict
% % $\CalE$ to be \emph{minimal}, i.e.\ there does not exist any other explanation
% % $\CalE' \subseteq \CalA$ for $\CalO$ such that $\CalE' \subseteq \CalE$.
% % 
% % Among the minimal explanations, it is possible that some of them entail a certain formula $F$ while others do not. There exist two strategies to determine whether $F$ is a valid conclusion in such cases. $F$ follows \emph{credulously}, if it is entailed by at least one explanation. $F$ follows \emph{skeptically}, if it is entailed by all explanations.
% % Due to previous results on modeling human reasoning~\cite{btg:2015:h},
% % skeptical abduction seems to be adequate.
% % 
% % Here, observations, are specified as
% % $\mathcal{O}_\mathcal{P} = \{ A \mid A \leftarrow \top \in \Def(A, \CalP) \}$. Usually, this set is further restricted by
% % considering only facts that result from the application of certain principles.
% % %\leftarrow \top \in def(A, \mathcal{P}) \land (A \leftarrow B_1 \land \dots \land B_n) \i%n def(A, \mathcal{P}) \},
% % The idea is to find an explanation for each observation $A \in
% % \CalO_\CalP$ after the fact $A \leftarrow \top$ has been removed from
% % $\Ground\CalP$. 



\section{Cognitive Principles}\label{sect:representing}

Eight principles for quantified statements have been developed in~\cite{cogsci:2017,costa:dietz:Hoelldobler:2017,declare:2017}.
In the following, the programs are
specified using the predicates $y$ and $z$, where~$yz$ can be replaced by $ab$, $ba$, $cb$, or $bc$ and depend on the figures shown in Table~\ref{t:2}. First, we introduce the basic principles that are assumed for all premises when applicable.
In Section~\ref{sect:advanced}, we present the advanced principles, which will 
later be used to distinguish between the clusters of reasoners.
% We indicate the cognitive principles in the text by specifying them in brackets.
A complete specification of the logic programs with respect to the moods according to all principles can be found in~\cite{declare:2017}.

\subsection{Basic Principles}   \label{sect:basic}

\subsubsection{Conditional Representation (\conditionals) and License for Inference (\licenses)}\label{sect:condli}
Consider \textit{All y are z} ({\MA yz}), which in CL, is represented as $\forall X (y(X) \rightarrow z(X))$.
Similarly, we assume that humans also understand this statement as a conditional sentence (\conditionals).
However, it seems that humans understand them defeasibly. According to Stenning and van Lambalgen~\cite{stenning:vanlambalgen:2008}, they assume a \textit{license for the inference} and might allow exceptions to the conditional (\licenses). For \MA yz this can be 
explicitly expressed by 
\textit{all $y$ that are not abnormal are $z$}.
% \\
% Accordingly, $z(X)$ is conjoint with an abnormality predicate: 
% \[\forall X (y(X) \wedge \neg \Ab(X) \rightarrow z(X)).\] 
By default, we assume that \textit{nothing is abnormal}. % with respect to $X$}:
% \[
% \Ab(X) \leftarrow \bot.
% \]

\subsubsection{Existential Import (\import)}

Furthermore, according to Grice~\cite{grice1975}, humans have a pragmatic understanding of quantifiers in the language.
For instance, in natural language, we normally do not quantify over things that do not exist.
Consequently, \textit{All $z$} implies \textit{Some $z$ exist}.
This is known as the \textit{Gricean implicature} or the \textit{existential import} (\import),
which also corresponds to the Aristotelian interpretation~\cite{parry:1991} and is assumed by several cognitive theories (cf.~\cite{johnsonlaird:1983,Rips1994}). 
%  Likewise,~\cite{stenning:vanlambalgen:2008} have shown that humans
% require existential import for a conditional to be true.
Together with the introduced cognitive principles in Section~\ref{sect:condli}, the logic program representing \MA yz, $\CalP_{Ayz}$, is as follows:
\begin{align} %{rll@{\hspace{\hfill}}r}
z(X) & \leftarrow  y(X) \wedge \neg \Ab_{yz}(X). 
\tag{\conditionals\ \&\ \licenses} \\
\Ab_{yz}(X) & \leftarrow \cxt(z'(X)). \tag{\licenses}\\
y(o) & \leftarrow \top. \tag{\import}
\end{align}
Note that $z'(X)$ appears within the $\cxt$ operator in the body of the second clause, where $z'$ represents the negation of $z$:\footnote{$z'(X)$ will be introduced by the \footnotesize{\textsf{transformation}} principle in Section~\ref{sect:dneg}.}
Nothing is abnormal, unless the negation of $z$ is true.
% , then $\Ab_{yz}(X)$ is false, for which then, given the first clause in $\CalP_{Ayx}$, $z(o)$ is true, if $y(o)$ is true. Note that, if $z'(X)$ would be written instead of $\cxt(z'(X))$, in the case where $\cxt(z'(o))$ is unknown, $z(o)$ could not be concluded, even if $y(o)$ would be true. This is due to the fact that negation in WCS is understood as strong negation.\footnote{This is different in other logic programming approaches, i.e.\ well-founded semantics.}
% \end{align}
% \]
We obtain \quad $\CalM_{\CalP_{Ayz}} = \langle \{y(o),z(o)\}, \{\Ab_{yz}(o)\} \rangle$.


\subsubsection{Negation by Transformation (\transformation) and \\No Double Negation (\dnegation)} \label{sect:dneg}

Consider the statement \textit{No y is z.} ({\ME yz}):
Similarly to the previous case, \licenses\ and \import\ apply here
as well.
The logic programs specified in Subsection~\ref{sect:wcs} 
do not allow heads of clauses to be negative literals. Therefore, a negative conclusion~$\neg p(X)$ is represented
by introducing an auxiliary formula $p'(X)$ together with the clause $p(X) \leftarrow \neg p'(X)$ and the integrity constraint $\udf \leftarrow p(X) \wedge p'(X)$ (\transformation). This is a widely used technique in Logic Programming (e.g. choice rules in~\cite{GebserKaminskiKaufmannSchaub12}). 
Another principle that we introduce is the \emph{no double negation} principle: Under the Weak Completion Semantics, a positive conclusion can be derived from double negation appearing in two conditionals. However, it seems that humans normally do not reason in such a way~\cite{khemlani:2012}. Hence, we block them with the help of abnormalities (\dnegation). $\CalP_{Eyz}$, the logic program representing \ME yz is as follows:
% \footnote{For simplicity, we omitted to list the clauses that are introduced by the (\converse) principle in $\CalP_{Eyz}$: The same clauses just switching $y$ and $z$ are added and thus also the fact $z(o_2) \leftarrow \top$ for another object $o_2$.
% The entire program can be found in~\cite{declare:2017}.}
\begin{align}
 z'(X) & \leftarrow y(X) \wedge \neg \Ab_{ynz}(X).  \tag{\conditionals\ \&\ \transformation\ \&\  \licenses} \\
\Ab_{ynz}(X)  & \leftarrow \cxt(z(X)).\tag{\licenses} \\
z(X) & \leftarrow \neg z'(X) \wedge \neg \Ab_{nzz}(X). \tag{\transformation\ \&\ \licenses} \\
y(o_1) & \leftarrow \top. \tag{\import} \\
\Ab_{nzz}(o_1) & \leftarrow \bot. \tag{\licenses\ \&\  \dnegation}
\end{align}
As we now have both $z(X)$ and $z'(X)$ in the head of some clause, we need to add the 
following integrity constraints: $\udf \leftarrow z(X) \wedge
z'(X)$. We obtain 
\[\CalM_{\CalP_{Eyz}} = \langle \{y(o_1), z'(o_1)\},\{\Ab_{ynz}(o_1), \Ab_{nzz}(o_1),  z(o_1)\}
\rangle.
\]


\subsubsection{Converse Implication (\converse) and \\ Unknown Generalization (\unknownGen)}
Consider the statement \textit{Some y are z.} ({\MI yz}), which in CL, is represented as $\exists X (y(X) \wedge z(X))$
and which is semantically equivalent to $\exists X(z(X) \wedge y(X))$ (\converse).
As shown by Khemlani and Johnson-Laird~\cite{khemlani:2012}, humans seem to
distinguish between \textit{some y are z} and \textit{all y are z} (\unknownGen).
Accordingly, if we only observe that an object $o$ belongs to $y$ and $z$ then we do not want to conclude both, \textit{some y are z} and \textit{all y are z}.
In order to distinguish between \textit{some y are z} and \textit{all y are z}, we introduce the following principle:  If we know that \textit{some y are z}, then there must not only be an object~$o_1$ which belongs to $y$ and $z$, but there must be another object $o_2$ which belongs to
$y$ and which does not belong to~$z$.
$\CalP_{Iyz}$, the logic program representing \MI yz, is as follows:\footnote{For simplicity, here we omit the clauses introduced by ({\footnotesize\textsf{converse}}). They can be found in~\cite{declare:2017}.}
\begin{align}
z(X) & \leftarrow y(X) \wedge \neg \Ab_{yz}(X). & \tag{\conditionals\ \&\ \licenses}\\
\Ab_{yz}(o_1) & \leftarrow  \bot. & \tag{\unknownGen\ \& \ \licenses}\\
y(o_1) & \leftarrow  \top. & \tag{\import} \\
y(o_2) & \leftarrow  \top. & \tag{\unknownGen} 
\end{align}
Consider the second clause: Different to the universal moods, \ME\ and \MA, $\Ab_{yz}$ is only false for $o_1$, which in turn implies that 
$\Ab_{yz}$ is unknown for all other objects in the program.
We obtain $\CalM_{\CalP_{Iyz}}= \langle \{y(o_1), y(o_2), z(o_1)\}, \{\Ab_{yz}(o_1)\} \rangle$.
One should observe that  $\Ab_{yz}(o_2)$ is an unknown assumption in
$\CalP_{Iyz}$ and,
hence, $\CalM_{\CalP_{Iyz}}(z(o_2)) = \udf$.

Consider the statement \textit{Some y are not z} ({\MO yz}), which in CL is represented as $\exists X (y(X) \wedge \neg z(X)$.
As in the previous cases, \conditionals, \licenses, and \unknownGen are assumed.
As the conclusion is again a negation of a claim, \transformation\ and
\dnegation\ apply as well. $\CalP_{Oyz}$, the logic program representing \MO yz is as follows:
\begin{align}
z'(X) & \leftarrow y(X) \wedge \neg \Ab_{ynz}(X). & \tag{\conditionals\ \&\ \transformation \ \&\ \licenses} \\
\Ab_{ynz}(o_1) & \leftarrow \bot. & \tag{\licenses\ \& \unknownGen} \\
z(X) & \leftarrow  \neg z'(X) \wedge \neg \Ab_{nzz}(X). &
\tag{\licenses\ \&\ \transformation} \\
y(o_1) & \leftarrow \top. & \tag{\import} \\
y(o_2) & \leftarrow \top. & \tag{\unknownGen} \\
\Ab_{nzz}(o_1)& \leftarrow \bot. & \tag{\licenses\ \&\ \dnegation}\\
\Ab_{nzz}(o_2)& \leftarrow \bot. & \tag{\licenses\ \&\ \dnegation}
\end{align}
We have to add the integrity constraint $U \leftarrow z(X) \wedge
z'(X)$
and obtain \[\CalM_{\CalP_{Oyz}} = \langle \{y(o_1), y(o_2), z'(o_1)\},\{\Ab_{ynz}(o_1),
\Ab_{nzz}(o_1), \Ab_{nzz}(o_2), z(o_1) \} \rangle.\]


\subsection{Advanced Principles}\label{sect:advanced}

The principles introduced here are assumed not to be applied by all humans, 
but characterize the reasoning clusters (c.f.\ Section~\ref{sect:aggregateddata}).


\subsubsection{Contraposition (\contraposition)}

In FOL, the conditional statement $\forall (X) (a(X) \rightarrow b(X))$, is logically
equivalent to its contrapositive, $\forall (X) (\neg b(X) \rightarrow \neg a(X))$.
% This
% contraposition also holds for the syllogistic moods~\MA\ and~\ME. 
There is
evidence that some of the participants might directly or indirectly 
apply the contrapositive (\contraposition) within a reasoning task (cf.~\cite{rips:1994,mentallogic:1994}).
Consider again $\CalP_{Ayz}$, the logic program representing \MA yz
from Section~\ref{sect:basic}. The clauses that represent the contraposition principles are as follows:
\begin{align}
y'(X) & \leftarrow  \neg z(X) \wedge \neg \Ab_{zy}(X). & \tag{\conditionals\ \&\ \licenses\ \&\ \contraposition\ } \\
\Ab_{zy}(X) & \leftarrow \cxt(y(X)). & \tag{ \licenses\ \&\ \contraposition}\\
y(X) & \leftarrow  \neg y'(X) \wedge \neg \Ab_{nyy}(X). & \tag{\transformation\ \&\ \licenses\ \&\ \contraposition} 
\end{align}
The application of \contraposition\ for the \ME\ mood,
\textit{No y is z}, yields \textit{No z is y}, and consists of the following clauses:
\begin{align}
y'(X) & \leftarrow z(X) \wedge \neg \Ab_{zny}(X). & \tag{\conditionals\ \&\ \licenses\ \&\ \contraposition\ } \\
\Ab_{zny}(X) & \leftarrow \cxt(y(X)). &\tag{ \licenses\ \&\ \contraposition}\\
y(X) & \leftarrow \neg y'(X) \wedge \neg \Ab_{nyy}(X). & \tag{\transformation\ \&\ \licenses\ \&\ \contraposition} 
% z(o_2) \leftarrow \top. & (\contraposition\  \&\ \import) \\
% \Ab_{nyy}(o_2) \leftarrow \bot. & (\contraposition\  \&\ \licenses\ \&\ \dnegation)
\end{align}


% Ignoring unknown information in conditionals is known as
% negation as failure~\cite{clark:1978} in logic programming.


\subsubsection{Generalization (\negFailure)}  %\label{sect:naf}

If all principles introduced so far are applied to a pair of existentially quantified premises
(\MI or \MO), then the only object about which an inference can be made is the one
resulting from \import\ with respect to the premise it introduced (i.e.\ only inferences about the properties 
of the object about $a$ and $b$ or
$b$ and $c$). 
% This is because the
% abnormality (\licenses) and according to (\unknownGen)
%  has to be false only for the object introduced by (\import), but unknown
% for other objects.
There is, however, evidence that some humans still draw conclusions about $a$ and $c$ in such
circumstances, i.e.\ they seem to generalize (universally quantify) the existential 
premise with respect to other objects (\negFailure).
Consider again $\CalP_{I yz}$ of Section~\ref{sect:basic}, where $\Ab_{yz}$ is only false for 
$o_1$. In case \negFailure\ applies, then $\Ab_{yz}(o_1) \leftarrow \bot$
in $\CalP_{I yz}$ is replaced by the following two clauses:
\begin{align}
\Ab_{yz}(X) & \leftarrow \cxt(z'(X)). & \tag{\licenses\ \&\ \negFailure} \\
\Ab_{yz}(o_2) & \leftarrow \udf. & \tag{\licenses\ \&\ \negFailure} 
\end{align}
Similar to the case of $\CalP_{Ayz}$ for the representation of the basic principles of \MA yz,
the $\cxt$ operator in the body of the first clause is false for all $o$
for which $z'(o)$ is unknown. This clause transforms the representation for the existential quantification over \MI yz in $\CalP_{Iyz}$
into a universal one. In order to maintain \unknownGen, the second clause is added:
$\Ab_{yz}(o_2) \leftarrow \udf$ guarantees that $\Ab_{yz}(o_2)$ stays unknown, even though $\cxt(z'(o_2))$ might be false.
Recall that programs are considered under their weak completion, i.e.\ if $\cxt(z'(o_2))$ is false, then
\[\Ab_{yz}(o_2) \leftrightarrow (\cxt(z'(o_2)) \vee \udf)
 \equiv \Ab_{yz}(o_2) \leftrightarrow \udf.\]
%  is semantically equivalent to $\Ab_{yz}(o_2) \leftrightarrow \udf$ if $\cxt(z'(o_2))$ is false.}




\subsubsection{Search for Alternative Conclusions (\abduction)}

When participants are faced with \NVC\ (\textit{No valid conclusion}), they might not want to accept this conclusion and proceed to check whether there exists unknown information that might be relevant. This information may be explanations about
facts introduced either from an existential import or from unknown generalization. 
We use the first as source for observations, as they are used directly to infer new information  (\abduction). Consider the following pair of syllogistic premises: 
\begin{align}
 \emph{All artists are bakers. } & \emph{Some bakers are not chemists.} \tag{AO1} 
\end{align}
Under CL, \textit{No valid conclusion} follows. However, 62\;\% of the participants concluded \textit{Some artists are not chemists}.
This conclusion can be modeled by abductive reasoning, where the system searches for explanations
for a given observation. The second premise states that some bakers exist, i.e.\ the observation is $\CalO = \{b(o_2) \}$. $\CalO$ can be explained by the first premise, namely that
this baker is an artist. Under WCS, this reasoning procedure was formalized by means of skeptical abduction.
The interested reader is referred to~\cite{cogsci:2017} for 
formal abductive approach for human syllogistic reasoning in this paper.



\subsection{Entailment of Conclusions}\label{sect:entail}

We specify when $\MP$ entails a conclusion with respect to $yz$:\footnote{$yz$ can be replaced by $ac$ or~$ca$.}
\begin{description}
\item[\MA yz (all)] $\CalP \models Ayz$ iff there exists an object $o$ such that $\CalP \ModelsWCS
y(o)$ and for all objects $o$ we find that if $\CalP \ModelsWCS y(o)$ then $\CalP \ModelsWCS z(o)$.
\item [\ME yz (no)]
$\CalP \models Eyz$ iff there exists an object $o_1$ such that $\CalP \ModelsWCS
y(o_1)$ and for all objects $o_1$ we find that if $\CalP \ModelsWCS y(o_1)$ then
$\CalP \ModelsWCS \neg z(o_1)$.\footnote{
When the contraposition principles applies, then additionally 
the following condition needs to hold as well: There exists an object $o_2$ such that $\CalP \ModelsWCS
z(o_2)$ and for all objects $o_2$ we find that if $\CalP \ModelsWCS z(o_2)$ then
$\CalP \ModelsWCS \neg y(o_2)$.}
\item[\MI yz (some)]
$\CalP \models Iyz$ iff there exists an object $o_1$ such that $\CalP \ModelsWCS
y(o_1) \wedge z(o_1)$ and there exists an
object $o_2$ such that $\CalP \ModelsWCS y(o_2)$ and
$\CalP \not\ModelsWCS z(o_2)$ and
% there exists an object $o_3$ such that $\CalP \ModelsWCS
% z(o_3) \wedge y(o_3)$ and 
there exists an
object $o_3$ such that $\CalP \ModelsWCS z(o_3)$ and
$\CalP \not\ModelsWCS y(o_3)$.
\item[\MO yz (some are not)]
$\CalP \models Oyz$ iff there exists an object $o_1$ such that $\CalP \ModelsWCS
y(o_1) \wedge \neg z(o_1)$ and there exists an
object $o_2$ such that $\CalP \ModelsWCS y(o_2)$ and
$\CalP \not\ModelsWCS \neg z(o_2)$.
\item[\NVC\ (no valid conclusion)] iff none of the previous conclusions can be derived.
\end{description}
For \MI yz (some) the conversion is explicitly specified, motivated by the \converse\ principle.
By requiring for \MI yz (\MO yz, resp.) the existence of an object with the first property $y$,
for which the second property $z$ has to be either either false or unknown (true or unknown, resp.),
\MA yz and \MI yz, (\ME yz and \MO yz, resp.), exclude each other.

\section{Evaluation} \label{sect:aggregateddata}



\subsection{On aggregated Data}


As illustration, let us consider \mbox{\MI\ME 2} from the introduction: 
\begin{align}
 \emph{Some bakers are artists. } & \emph{No chemists are bakers.} \tag{\MI\ME 2} 
\end{align}
The majority of participants in experimental studies
concluded \emph{No chemists are artists} (29\;\%), \emph{No valid conclusion} (27\;\%)
and \emph{Some artists are not chemists} (16\;\%).
In order to account for the individual differences among participants, three clusters on principles and two on heuristic strategies have been suggested
in~\cite{declare:2017}. The two heuristic strategies are the atmosphere bias~\cite{woodworth:sells:1935} and the matching bias~\cite{Wetherick1995}.\footnote{
The interested reader is referred to~\cite{declare:2017} for details on their application in syllogistic reasoning. 
}
The three reasoning clusters have been characterized by the following principles:\footnote{{\footnotesize\textsf{basic}} refers to the set of basic principles introduced in Subsection~\ref{sect:basic}.} (1) \basic, \converse\ for \MI, and \abduction,
    (2) \basic, \converse\ for \MI, and \negFailure, and
    (3) \basic, \converse\ for \MI\ and \ME, and \contraposition\ for~\MA.
This approach achieved a match of 92\;\% with the aggregated data reported in~\cite{khemlani:2012}, and was the highest result compared to the presented approaches.

Table~\ref{tab:cluster} summarizes the three clusters together with their conclusions for \MI\ME 2.
\begin{table}
\begin{center}
\begin{tabular}{@{\hspace{0cm}}l@{\hspace{0.9cm}}c@{\hspace{0.9cm}}c@{\hspace{0.9cm}}cGG}
Principle & (1) & (2) & (3) & Heuristic 1 & Heuristic 2 \medskip \\ \midrule  
\conditionals & $\checkmark$   & $\checkmark$ & $\checkmark$ & - & - \smallskip\\
\licenses & $\checkmark$ & $\checkmark$       & $\checkmark$ & - & -   \smallskip \\
\import  & $\checkmark$  & $\checkmark$       & $\checkmark$ & - & - \smallskip\\
\unknownGen & $\checkmark$  & $\checkmark$    & $\checkmark$ & - & -     \smallskip\\
\converse\ for \MI & $\checkmark$  & $\checkmark$      & $\checkmark$ & - & -   \smallskip\\
\abduction &  $\checkmark$ & -             & -            & - & - \smallskip\\
\negFailure &  - &  $\checkmark$  & -             & - & - \smallskip\\
\converse\ for \ME & -  & -  & $\checkmark$      & - & -   \smallskip\\
\contraposition\ for \MA & - & -  & $\checkmark$          & - & - \smallskip\\\midrule
% matching   &  -            & -             & -            & $\checkmark$ & -   \smallskip\\
% biased fig 1 & -     & -             & -            & -  & $\checkmark$ \medskip \\ \midrule
 &  $\Downarrow$ & $\Downarrow$ &  $\Downarrow$ & $\Downarrow$  &  $\Downarrow$ \\
 &  \NVC &  \textit{Some artists are not chemists}  &   \textit{No chemists are artists} \\
%            &       & \textit{} & \textit{}
\end{tabular}
\caption{Cognitive principles and the corresponding entailments for \MI\ME 2.
\label{tab:cluster}}
\end{center}
\end{table}
The corresponding logic program $\Pie$ that represents the basic cluster, consists of the following clauses:
\begin{align}
a(X) & \leftarrow b(X) \wedge \neg \Ab_{ba}(X). & \tag{\conditionals\ \&\ \licenses}\\
\Ab_{ba}(o_1) & \leftarrow  \bot. & \tag{\licenses\ \& \unknownGen}\\
b(o_1) & \leftarrow  \top. & \tag{\import} \\
b(o_2) & \leftarrow  \top. & \tag{\unknownGen}
\medskip \\
b(X) & \leftarrow a(X) \wedge \neg \Ab_{ab}(X). & \tag{\conditionals\ \&\ \licenses\ \&\ \converse}\\
\Ab_{ab}(o_3) & \leftarrow  \bot. & \tag{\unknownGen\ \& \ \licenses\ \&\ \converse}\\
a(o_3) & \leftarrow  \top. & \tag{\import\ \&\ \converse} \\
a(o_4) & \leftarrow  \top. & \tag{\unknownGen\ \& \converse} 
\medskip \\
% \end{align}
% \begin{align}
b'(X) & \leftarrow c(X) \wedge \neg \Ab_{cnb}(X).  \tag{\licenses\ \&\ \transformation} \\
\Ab_{cnb}(X)  & \leftarrow \cxt(b(X)).\tag{\licenses} \\
b(X) & \leftarrow \neg b'(X) \wedge \neg \Ab_{nbb}(X). \tag{\licenses\ \&\ \transformation} \\
c(o_5) & \leftarrow \top. \tag{\import} \\
\Ab_{nbb}(o_5) & \leftarrow \bot. \tag{\licenses\ \&\  \dnegation}
\end{align}
We obtain $\CalM_\Pie = \langle I^\top, I^\bot \rangle$, where 
\[
\begin{array}{lll}
 I^\top & = & \{ b(o_1), a(o_1), b(o_2), a(o_3), b(o_3),  a(o_4), c(o_5), b'(o_5), \Ab_{cnb}(o_1), \Ab_{cnb}(o_2), \Ab_{cnb}(o_3)  \}, \\
I^\bot & = & \{ \Ab_{ba}(o_1), \Ab_{ab}(o_3), \Ab_{cnb}(o_5), \Ab_{nbb}(o_5), \Ab_{cnb}(o_4) \}, 
\end{array}
\]
for which according to the entailment of conclusions as defined in Section~\ref{sect:entail}, \textit{No valid conclusion}
follows.

Applying the contraposition principle to \textit{No chemists are bakers} yields \textit{No bakers are chemists}.
$\Piecontra$ consists of the following clauses:
\begin{align*}
\Pie \cup \{ c'(X) & \leftarrow b(X) \wedge \neg \Ab_{bnc}(X), & \tag{\basic\ \&\ \licenses\ \&\ \transformation\  \&\  \contraposition} \\
\Ab_{bnc}(X) & \leftarrow \cxt(c(X)), & \tag{\contraposition\  \&\ \licenses} \\
c(X) & \leftarrow \neg c'(X) \wedge \neg \Ab_{ncc}(X), &  \tag{\licenses\ \&\ \transformation\  \&\  \contraposition} \\
b(o_6) & \leftarrow \top, & \tag{\import\ \&\ \contraposition} \\
\Ab_{ncc}(o_6) & \leftarrow \bot  \} & \tag{\licenses\ \& \ \dnegation\ \&\ \contraposition}
\end{align*}
We obtain $\CalM_{\Piecontra} = \langle I^{\top'}, I^{\bot'} \rangle$ where 
\[
\begin{array}{lllll}
 I^{\top'} & = & I^\top & \cup & \{ b(o_6), c'(o_3), c'(o_1), c'(o_2), c'(o_6), \Ab_{bnc}(o_5), \Ab_{cnb}(o_6) \}, \\
I^{\bot'} & =  & I^\bot  & \cup & \{  \Ab_{bnc}(o_i) \mid i \in \{1,2,3,4,6 \} \} 
\\ &&& \cup  &
\{ c(o_i), b'(o_1) \mid i \in \{ 1, 2, 3, 6 \} \} \cup \Ab_{cnb}(o_3), \Ab_{ncc}(o_6) \}.
\end{array}
\]
\textit{Some artists are not chemists} is entailed by $\CalM_{\Piecontra}$ because 
$a(o_1)$ is true and $c(o_1)$ is false and $a(o_4)$ is true, but $c(o_4)$ is unknown.

Applying the generalization principle to \textit{Some bakers are artists} and its converse,
yields to \textit{All bakers are artists} and \textit{All artists are bakers}, respectively.
$\Piegen$ consists of the clauses in $\CalP_{IE2}$ and the following ones:
\begin{align}
\Pie \cup \{  \Ab_{ba}(X) & \leftarrow \cxt(a'(X)),  \Ab_{ba}(o_2) \leftarrow \udf, & \tag{\basic\ \&\ \licenses\ \&\ \negFailure} \\
  \Ab_{ab}(X) & \leftarrow \cxt(b'(X)),   \Ab_{ab}(o_4) \leftarrow \udf \}  & \tag{\licenses\ \&\ \negFailure\ \&\ \converse}
\end{align}
We obtain $\CalM_{\Piegen}  = \langle I^{\top''}, I^{\bot''} \rangle$ where 
\[
\begin{array}{lllll}
 I^{\top''} & = & I^\top & \cup & \{\Ab_{ab}(o_5)  \}, \\
I^{\bot''} & = & I^\bot & \cup & \{ a(o_5),  b(o5) \cup \{ \Ab_{ab}(o_i) \mid i \in \{ 1,2\}\} \\
& & & \cup &
\{ \Ab_{ba}(o_i) \mid i \in \{3, 4, 5 \} \} \cup \{ \Ab_{cnb}(o_3) \mid i \in \{1,2,3 \} \}.
\end{array}
\]
\textit{No chemists are artists} is entailed by $\CalM_{\Piegen}$, because 
for all objects which have the property $c$, (which only applies for $o_5$), they do not have the property.

On aggregated data, the clusters performed fairly well with a match of 92\;\%. 
We are interested whether these clusters also correspond to the reasoning patterns observed among individuals. 
For this purpose, we investigated the non-aggregated dataset provided by Johnson-Laird and Khemlani~\cite{khemlani:2016} consisting of the responses of 20 participants for two weeks,
i.e.\ the participants carried out the task twice, with a break of one week. 
The results are two-fold:
\begin{description}
 \item[Aggregated evaluation] Most of the majority's answers can be explained by the clusters' predictions. In the first week 
 the majority's conclusion for 63 out of 64 pair of syllogistic premises and in the second week
 all majority's conclusions were predicted.
\item[Non-aggregated evaluation] None of the clusters fitted any individual participants' response pattern, i.e.\ none of the participants belonged to a certain cluster.
\end{description}



Given this sobering outcome with respect to the non-aggregated evaluation, 
we ignored the clusters, and investigated whether parallels between the applied cognitive principles could be observed: \basic, \contraposition, \abduction\ and \negFailure.
Table~\ref{tab:cogprinc} provides an overview of the predicted conclusions according to these principles.\footnote{Note that any advanced principle also includes the logic program clauses specified for \textsf{basic}.}
The predictions of the principles naturally raises a few questions on the plausibility of 
the underlying assumptions:
(i) Do participants use the same reasoning strategy throughout the same reasoning task, i.e.\ 
can their conclusions be 
explained by  the same principles throughout the whole reasoning task, or do they switch strategies within one reasoning task?
(ii) Is the list of principles in Section~\ref{sect:representing} complete, i.e.\ can the conclusions of all reasoners be covered
 by these principles? and
(iii) Do participants learn, i.e.\ do they improve their logical ability by doing the same reasoning task again?
We address these questions in the following.

\clearpage
\begin{table}[t]\betweentinyandsmall\centering
\newcolumntype{a}{>{\columncolor{white}}c@{\hspace{2mm}}}
\newcolumntype{b}{>{\columncolor{gray!40}}c@{\hspace{2mm}}}  
\captionsetup{font=footnotesize}
 \begin{tabular}{llllllllll}
 \toprule
	 & \MA ac & \ME ac & \MI ac & \MO ac & \MA ca & \ME ca & \MI ca & \MO ca & \NVC \\ \midrule 
	 
\rowcolor{lightgray} 
\MA\MA1  & \basicc&        &        &        &        &        &        &        &      \\ 
\MA\MA2  &        &        &        &        & \basicc&        &        &        & \\ \rowcolor{lightgray} 
\MA\MA3  &        &        &        &        &        &        &        &        & \basicc \\
\MA\MA4  & \basicc&        &        &        &\basicc &        &        &        & \\  \rowcolor{lightgray}  %\midrule 

\MA\MI1  &        &        & \abdc  &        &        &        & \abdc  &        & \basicc \\
\MA\MI2  &        &        & \basicc&        &        &        & \basicc \\  \rowcolor{lightgray} 
\MA\MI3  &        &        & \abdc  &        &        &        & \abdc  &        & \basicc \\
\MA\MI4  &        &        & \basicc&        &        &        & \basicc&        & \\%\midrule 

\rowcolor{lightgray} 
\MA\ME1  &        & \basicc&        &        &        &        &        &        & \\ %\rowcolor{lightgray} 
\MA\ME2  &        &\contrac&        &        &        & \basicc&        &        & \\ \rowcolor{lightgray} 
\MA\ME3  &        &\contrac&        &        &        &\contrac&        &        & \basicc \\ %\rowcolor{lightgray} 
\MA\ME4  &        & \basicc&        &        &        &\contrac&        &        &  \\  \rowcolor{lightgray}  %\midrule 

\MA\MO1  &        &        &        &\abdc   &        &        &        &        & \basicc \\
\MA\MO2  &        &        &        &        &        &        &        &\basicc \\  \rowcolor{lightgray} 	
\MA\MO3  &        &        &        &        &        &        &        &\contrac&\basicc \\
\MA\MO4  &        &        &        &\basicc \\ %\midrule 

\rowcolor{lightgray} 
\MI\MA1  &        &        &\basicc &        &        &        &\basicc &        & \\ % \rowcolor{lightgray} 
\MI\MA2  &        &        &\abdc   &        &\genc   &        & \abdc  &        & \basicc \\ \rowcolor{lightgray} 
\MI\MA3  &        &        & \abdc  &        &        &        & \abdc  &        & \basicc \\ % \rowcolor{lightgray} 
\MI\MA4  &        &        &\basicc &        &        &        &\basicc &        &  \\  %\midrule
 \rowcolor{lightgray} 
\MI\MI1  &        &        &\genc   &        &        &        & \genc  &        & \basicc \\
\MI\MI2  &        &        &\genc   &        &        &        & \genc  &        & \basicc \\ \rowcolor{lightgray} 
\MI\MI3  &        &        &\genc   &        &        &        & \genc  &        & \basicc \\
\MI\MI4  &        &        &\genc   &        &        &        & \genc  &        & \basicc \\ %\midrule 

\rowcolor{lightgray} 
\MI\ME1  &        &        &        &\basicc &        &        &        &        & \\ %\rowcolor{lightgray} 
\MI\ME2  &        &        &        &\contrac&        & \genc  &        &        & \basicc \\\rowcolor{lightgray} 
\MI\ME3  &        &        &        &\contrac&        & \genc  &        &        & \basicc  \\% \rowcolor{lightgray} 
\MI\ME4  &        &        &        & \basicc&        &        &        &        & \\ %\midrule 
 \rowcolor{lightgray} 
\MI\MO1  &        &        &        & \genc  &        &        &        &        & \basicc \\
\MI\MO2  &        &        &        & \genc  &        &        &        &        & \basicc \\  \rowcolor{lightgray} 
\MI\MO3  &        &        &        & \genc  &        &        &        &        & \basicc \\ 
\MI\MO4  &        &        &        & \genc  &        &        &        &        & \basicc \\%\midrule

\rowcolor{lightgray} 
\ME\MA1  &        & \basicc&        &        &        &        &        &        & \\ %\rowcolor{lightgray} 
\ME\MA2  &        &\contrac&        &        &        & \basicc&        &        &  \\\rowcolor{lightgray} 
\ME\MA3  &        &\contrac&        &        &        &\contrac&        &        & \basicc \\% \rowcolor{lightgray} 
\ME\MA4  &        &\contrac&        &        &        & \basicc&        &        &\\  %\midrule 
 \rowcolor{lightgray} 
\ME\MI1  &        & \genc   &        &	 &	  &	   &	    &	     & \basicc \\
\ME\MI2  &        &	       &	&	 &	  &	   &	    & \basicc \\ \rowcolor{lightgray}  
\ME\MI3  &        & \genc   &        &	 &	  &	   &	    &\contrac& \basicc \\
\ME\MI4  &        &	       &	&	 &	  &	   &	    & \basicc \\%\midrule

\rowcolor{lightgray} 
\ME\ME1  &        &	       &	&	 &	  &	   &	    &        &\basicc \\ %\rowcolor{lightgray} 
\ME\ME2  &        &	       &	&	 &	  &	   &	    &        &\basicc \\ \rowcolor{lightgray} 
\ME\ME3  &        &	       &	&	 &	  &	   &	    &        &\basicc \\ %\rowcolor{lightgray} 
\ME\ME4  &        &	       &	&	 &	  &	   &	    &        &\basicc \\%\midrule
 \rowcolor{lightgray} 
\ME\MO1  &        &	       &	&	 &	  &	   &	    &        &\basicc \\
\ME\MO2  &        &	       &	&	 &	  &	   &	    &        &\basicc \\ \rowcolor{lightgray} 
\ME\MO3  &        &	       &	&	 &	  &	   &	    &        &\basicc \\
\ME\MO4  &        &	       &	&	 &	  &	   &	    &        &\basicc \\ %\midrule

\rowcolor{lightgray} 
\MO\MA1  &        &            &	& \basicc&        &        &        &        & \\ %\rowcolor{lightgray} 
\MO\MA2  &        &	       &	&	 &	  &	   &	    &  \abdc &\basicc \\\rowcolor{lightgray} 
\MO\MA3  &        &	       &	&\contrac&	  &	   &	    &        &\basicc \\% \rowcolor{lightgray} 
\MO\MA4  &        &	       &	&	 &	  &	   &	    & \basicc& \\  %\midrule
 \rowcolor{lightgray} 
\MO\MI1  &        &	       &	&\genc   &	  &	   &	    &        &\basicc \\
\MO\MI2  &        &	       &	&	 &	  &	   &	    &\genc   &\basicc \\  \rowcolor{lightgray} 
\MO\MI3  &        &	       &	&\genc   &	  &	   &	    &        &\basicc \\ 
\MO\MI4  &        &	       &	&	 &	  &	   &	    &\genc   &\basicc\\ %\midrule 

\rowcolor{lightgray} 
\MO\ME1  &        &	       &	&        &	  &	   &	    &        &\basicc \\ %\rowcolor{lightgray} 
\MO\ME2  &        &	       &	&        &	  &	   &	    &        &\basicc \\\rowcolor{lightgray} 
\MO\ME3  &        &	       &	&        &	  &	   &	    &        &\basicc \\ %\rowcolor{lightgray} 
\MO\ME4  &        &	       &	&        &	  &	   &	    &        &\basicc \\ %\midrule
\rowcolor{lightgray} 
\MO\MO1  &        &	       &	&        &	  &	   &	    &        &\basicc \\
\MO\MO2  &        &	       &	&        &	  &	   &	    &        &\basicc \\\rowcolor{lightgray}  
\MO\MO3  &        &	       &	&        &	  &	   &	    &        &\basicc \\
\MO\MO4  &        &	       &	&        &	  &	   &	    &        &\basicc \\ \bottomrule
  \end{tabular}

\caption{\label{tab:cogprinc}Predictions according to the cognitive principles as specified in Section~\ref{sect:representing}.\betweentinyandsmall}
 \end{table}
 
\thispagestyle{empty}
\clearpage
\subsection{On non-aggregated Data} \label{sect:nonag}

A wide amount of cognitive theories for syllogistic reasoning have previously only be evaluated with respect to the aggregated data of psychological experiments.
Usually, as assessment, typical statistical evaluation criteria with respect to the experimental data 
 have been used (cf. root-mean-square error, Bayesian information criterion~\cite{schwarz1978}, Akaike information criterion~\cite{akaike1974new}).
 These measures are helpful as a general guideline, however, their major drawback is that they only compute the quantitative overall performance.
 In this section, we carry out a qualitative assessment of the suggested cognitive principles with respect to the individual differences
  among the participants.
% As already mentioned in the introduction, an issue with these evaluations is that these theories cannot account for individual differences. 
% Therefore, in this section, we will investigate an approach that accounts for non-aggregated data of participants in syllogistic reasoning. 

\subsubsection{mReasoner}

Johnson-Laird and Khemlani~\cite{khemlani:2016} proposed an approach to model the individual reasoning patterns of the participants, and 
showed that different participants seem to derive different conclusions.
Their approach is built on top of the mReasoner~\cite{khemlani:2013}, an extension of the mental model theory, first introduced by Johnson-Laird~\cite{johnsonlaird:1983}.
Reasoning depends on the construction and manipulation of mental models, based on 3 fundamental principles: 
(i) represent possibilities, (ii) principle of iconicity, and (iii) principle of dual processes.
 The procedure of mReasoner consists of two steps:
(1) The construction of a small set of models that represents the terms of the premises, and
(2) the call of a deliberative component to search for counterexamples to conclusions.
Johnson-Laird and Khemlani specified four parameters for the construction of a mental model and evaluated their
predictions based on 2 datasets, week 1 and week 2, for 20 participants.
The parameters for the construction of the mental models are (i) the 
size of a mental model, (ii) the probability of drawing conclusions only from the most common mental models, or from the complete set,
(iii) the probability to search for counter examples,
and (iv) the probability to weaken the conclusion.
The clustering with mReasoner was based on a particular clustering algorithm to determine the optimal number of clusters for their dataset.
The K-means clustering algorithm was used to divide participants into clusters and then
the mReaoner to simulated the clusters by choosing appropriate parameter settings.
Johnson-Laird and Khemlani~\cite{khemlani:2016} identified three different reasoning clusters: The intuitive cluster, the intermediate cluster and the deliberative cluster.
The approach is novel in the sense that until now no other cognitive theory tried to rigorously account for individual differences in human syllogistic reasoning. However, their approach seems to have two drawbacks. First, some of their 
parameters were of probabilistic nature, i.e.\ there is no explaining factor behind them. Second, the authors did not distinguish between the participants' 
responses in week 1 and week 2, but instead merged the responses. This might give a distorted picture, as participants could
have learned from week 1 to week 2.

% The clusters described in~\cite{declare:2017} seem not to correspond exactly to the individual participants. 
% and observed an interesting effect: Participants seemed to improve their logical ability when carrying out the task for the second time.
% Furthermore, participants seemed not to reason according to one specific cluster during the whole task. This puts forward
% two possible assumptions: Either the clusters in~\cite{declare:2017} were not well defined or people changed their reasoning techniques within the task. 
% % According to these new findings, we formulate the hypothesis that the clusters do not represent different human reasoners, but different stages of reasoning. Depending on the persistence and the previous training of the individual participants, they might be in different stages.
% taking as starting point the principles in~\cite{cogsci:2017}.
% This dataset consists of the individual conclusions of 20 participants, who carried out the same 
% task two times, with an interval of one week. 




 

\subsubsection{Individual Reasoning Patterns}

\begin{table}[htp]
\betweentinyandsmall\centering
\newcolumntype{a}{>{\columncolor{white}}c@{\hspace{2mm}}}
\newcolumntype{b}{>{\columncolor{gray!40}}c@{\hspace{2mm}}}  
\captionsetup{font=footnotesize}
\begin{tabular}{cccccccccccc}\toprule
Week & Part. & NVC & CL & \basicc & \contrac & \abdc & \genc & Changes & Cluster & Cluster in~\cite{khemlani:2016}  \\ \midrule 
\rowcolor{lightgray} 
1&0&$\downarrow$&$\downarrow$&$\downarrow$&$\downarrow$&$\downarrow$&$\uparrow$& & &  \\
\rowcolor{lightgray}  
2&0&$\downarrow$&$\leftrightarrow$&$\leftrightarrow$&$\uparrow$ $\uparrow$&$\downarrow$&$\downarrow$ &
\multirow{-2}{*}{$\uparrow$}  & \multirow{-2}{*}{INTUITIVE}& 
\multirow{-2}{*}{INTUITIVE} \\  

1&1&$\downarrow$&$\downarrow$&$\downarrow$&$\uparrow$&$\uparrow$& $\uparrow$& &&\\
2&1&$\uparrow$&$\uparrow$&$\uparrow$&$\leftrightarrow$&$\uparrow$&$\downarrow$& \multirow{-2}{*}{$\leftrightarrow$}&
\multirow{-2}{*}{INTUITIVE} &\multirow{-2}{*}{INTER}  \\ 

\rowcolor{lightgray} 
1&2&$\downarrow$&$\downarrow\downarrow$&$\downarrow\downarrow$&$\downarrow$&$\uparrow$&$\uparrow\uparrow$& && \\ \rowcolor{lightgray} 
2&2&$\downarrow$&$\downarrow$&$\leftrightarrow$&$\downarrow$&$\uparrow$&$\uparrow\uparrow$ & 
\multirow{-2}{*}{$\uparrow$} & \multirow{-2}{*}{INTUITIVE}&  \multirow{-2}{*}{INTUITIVE} \\

1&3&$\downarrow$&$\downarrow\downarrow$&$\downarrow\downarrow$&$\leftrightarrow$&$\uparrow$&$\leftrightarrow$ &&& \\
2&3&$\downarrow$&$\downarrow$&$\downarrow$&$\downarrow\downarrow$&$\uparrow$&$\leftrightarrow$ &
\multirow{-2}{*}{$\uparrow$} &\multirow{-2}{*}{INTUITIVE}& \multirow{-2}{*}{INTUITIVE}\\

\rowcolor{lightgray} 
1&4&$\uparrow$&$\uparrow$&$\uparrow$&$\uparrow\uparrow$&$\downarrow$&$\downarrow\downarrow$ &&& \\ 
\rowcolor{lightgray} 
2&4&$\uparrow$&$\uparrow$&$\uparrow$&$\uparrow\uparrow$&$\downarrow$&$\downarrow\downarrow$ &\multirow{-2}{*}{$\downarrow$}&
\multirow{-2}{*}{DEL}& \multirow{-2}{*}{DEL} \\

1&5&$\uparrow$&$\uparrow$&$\uparrow$&$\uparrow$&$\downarrow$&$\downarrow$ &&& \\
2&5&$\uparrow$&$\uparrow$&$\uparrow\uparrow$&$\uparrow\uparrow$&$\downarrow$&$\downarrow\downarrow$ &\multirow{-2}{*}{$\downarrow$}&
\multirow{-2}{*}{DEL}& \multirow{-2}{*}{DEL}\\

\rowcolor{lightgray} 
1&6&$\leftrightarrow$&$\leftrightarrow$&$\leftrightarrow$&$\downarrow$&$\leftrightarrow$&$\uparrow\uparrow$ &&&\\  \rowcolor{lightgray} 
2&6&$\leftrightarrow$&$\leftrightarrow$&$\downarrow$&$\uparrow$&$\uparrow\uparrow$&$\uparrow$ &\multirow{-2}{*}{$\uparrow$}&\multirow{-2}{*}{INTER}&\multirow{-2}{*}{INTER}\\

1&7&$\uparrow$&$\uparrow$&$\uparrow$&$\uparrow$&$\downarrow$&$\downarrow$ &&& \\
2&7&$\uparrow$&$\uparrow$&$\uparrow$&$\leftrightarrow$&$\downarrow$&$\downarrow\downarrow$ &\multirow{-2}{*}{$\downarrow$}&\multirow{-2}{*}{DEL} &\multirow{-2}{*}{DEL}\\

\rowcolor{lightgray} 
1&8&$\uparrow$&$\uparrow$&$\uparrow$&$\uparrow$&$\uparrow$&$\downarrow$ &&& \\
\rowcolor{lightgray} 
2&8&$\uparrow$&$\uparrow$&$\uparrow$&$\uparrow\uparrow$&$\uparrow$&$\downarrow\downarrow$ &\multirow{-2}{*}{$\downarrow$}&\multirow{-2}{*}{DEL}&\multirow{-2}{*}{DEL}\\

1&9&$\leftrightarrow$&$\leftrightarrow$&$\leftrightarrow$&$\leftrightarrow$&$\uparrow$&$\uparrow$ &&& \\
2&9&$\leftrightarrow$&$\uparrow$&$\uparrow$&$\leftrightarrow$&$\downarrow$&$\downarrow\downarrow$ &\multirow{-2}{*}{$\uparrow$} &\multirow{-2}{*}{INTER}& \multirow{-2}{*}{INTER}\\

\rowcolor{lightgray} 
1&10&$\uparrow$&$\uparrow$&$\uparrow$&$\uparrow$&$\downarrow$&$\downarrow\downarrow$ &&&\\
\rowcolor{lightgray} 
2&10&$\uparrow$&$\uparrow$&$\uparrow\uparrow$&$\uparrow$&$\downarrow$&$\downarrow\downarrow$ & \multirow{-2}{*}{$\downarrow$}&\multirow{-2}{*}{DEL}&\multirow{-2}{*}{DEL}\\

1&11&$\downarrow$&$\leftrightarrow$&$\uparrow$&$\uparrow$&$\uparrow$&$\uparrow$ &&&\\
2&11&$\downarrow$&$\downarrow$&$\downarrow$&$\downarrow$&$\uparrow$&$\uparrow$ &\multirow{-2}{*}{$\downarrow$}&\multirow{-2}{*}{INTER}&\multirow{-2}{*}{INTUITIVE}\\

\rowcolor{lightgray} 
1&12&$\leftrightarrow$&$\leftrightarrow$&$\leftrightarrow$&$\leftrightarrow$&$\downarrow$&$\leftrightarrow$ &&&\\
\rowcolor{lightgray} 
2&12&$\leftrightarrow$&$\leftrightarrow$&$\leftrightarrow$&$\downarrow$&$\uparrow$&$\uparrow$ &\multirow{-2}{*}{$\leftrightarrow$}&\multirow{-2}{*}{INTER}&\multirow{-2}{*}{INTER} \\

1&13&$\leftrightarrow$&$\downarrow$&$\leftrightarrow$&$\uparrow$&$\downarrow$&$\uparrow$ &&&\\
2&13&$\uparrow$&$\downarrow$&$\leftrightarrow$&$\downarrow$&$\uparrow$&$\downarrow$& \multirow{-2}{*}{$\uparrow$}&\multirow{-2}{*}{INTER}&\multirow{-2}{*}{INTER} \\

\rowcolor{lightgray} 
1&14&$\downarrow$&$\downarrow$&$\uparrow$&$\downarrow$&$\uparrow\uparrow$&$\uparrow\uparrow$ &&&\\
\rowcolor{lightgray} 
2&14&$\uparrow$&$\leftrightarrow$&$\uparrow\uparrow$&$\downarrow$&$\uparrow$&$\uparrow$ & \multirow{-2}{*}{$\downarrow$}&\multirow{-2}{*}{INTER}&\multirow{-2}{*}{INTER} \\

1&15&$\downarrow$&$\downarrow$&$\downarrow$&$\downarrow\downarrow$&$\downarrow$&$\uparrow$ &&&\\
2&15&$\downarrow$&$\downarrow$&$\downarrow$&$\downarrow$&$\uparrow\uparrow$&$\uparrow\uparrow$ &\multirow{-2}{*}{$\uparrow\uparrow$}&\multirow{-2}{*}{INTUITIVE}&\multirow{-2}{*}{INTUITIVE}\\

\rowcolor{lightgray} 
1&16&$\uparrow$&$\downarrow$&$\leftrightarrow$&$\downarrow$&$\uparrow$&$\downarrow$ &&& \\
\rowcolor{lightgray} 
2&16&$\downarrow$&$\downarrow$&$\leftrightarrow$&$\uparrow$&$\uparrow$&$\downarrow$&\multirow{-2}{*}{$\uparrow$}&\multirow{-2}{*}{INTER}&\multirow{-2}{*}{INTER}\\

1&17&$\downarrow$&$\downarrow$&$\downarrow$&$\downarrow$&$\uparrow\uparrow$&$\uparrow\uparrow$ &&&\\
2&17&$\leftrightarrow$&$\downarrow$&$\downarrow$&$\uparrow\uparrow$&$\uparrow$&$\uparrow\uparrow$ &\multirow{-2}{*}{$\uparrow$} &
\multirow{-2}{*}{INTUITIVE}& \multirow{-2}{*}{INTUITIVE}\\

\rowcolor{lightgray} 
1&18&$\uparrow$&$\uparrow$&$\uparrow$&$\uparrow$&$\downarrow$&$\downarrow\downarrow$ &&&\\
\rowcolor{lightgray} 
2&18&$\uparrow$&$\uparrow$&$\uparrow\uparrow$&$\downarrow$&$\downarrow$&$\downarrow\downarrow$ &\multirow{-2}{*}{$\downarrow$}& \multirow{-2}{*}{DEL} & \multirow{-2}{*}{DEL}\\

1&19&$\leftrightarrow$&$\downarrow$&$\uparrow$&$\downarrow$&$\downarrow$&$\downarrow$ &&&\\
2&19&$\leftrightarrow$&$\leftrightarrow$&$\downarrow$&$\downarrow$&$\downarrow$&$\downarrow$ &\multirow{-2}{*}{$\uparrow$}& \multirow{-2}{*}{INTUITIVE}& \multirow{-2}{*}{INTER}\\ \bottomrule
\end{tabular}
\caption{Overview of the individual participants' responses with respect to the specified criteria in week 1 and week 2.
$\uparrow$, $\downarrow$, and $\leftrightarrow$ denote above, below, and average, respectively.
\label{tab:individual}}
\end{table}

The non-aggregated data included a dataset of 20 participants,
for which the task was carried out twice, with a break of one week~\cite{khemlani:2016}. On average, the basic principles explained the results about 10\;\% better than CL.
 Generally, the participants who were strong in CL/ contraposition, drew fewer conclusions according to abduction/ generalization, and vice versa.
 The changes within participant's answers from one to the other week was high, with 52/64 on average. 
 From week 1 to week 2, an increase with respect to CL (19/20, 6\;\%), basic (18/20, 6\;\%), contraposition (15/20, 5\;\%)
 and a decrease on generalization (15/20, 8\;\%) and no change on abduction (10/20, 0\;\%) answers could be observed.

Table~\ref{tab:individual} provides an overview of the performance of of each participant (column 2), for week 1 and week 2 (column 1)
with respect to CL (column 4) and the cognitive principles (column 5, 6, 7, and 8).
NVC (column 3) denotes how frequently participants responded \textit{no valid conclusion} and Changes (column 9) denotes how
frequently participants changed their answers from week 1 to week 2. 
The symbols $\uparrow$, $\downarrow$, and $\leftrightarrow$ denote above, below, and average, respectively.

We identified three different clusters: 
The last two columns show the classification of the participants as deliberative, intuitive or intermediate. 
% The participants in the deliberative cluster here corresponds exactly to~\cite{khemlani:2016}. 
Our classification only differs wrt.\ Johnson-Laird and Khemlani's 
classification~\cite{khemlani:2016} for the intermediate and intuitive cluster with respect to three participants (1, 11 and 19).
A possible explanation for this difference could be that they
did not take into account the participants' changes between week 1 and week 2.


Two of the three clusters can be specified through a set of cognitive principles, and for one, no distinguishable pattern could be identified:
% The summary is shown in Table~\ref{tab:clusters}.
\begin{description}
 \item[Deliberative cluster] (6 out of 20 participants) is characterized by the participants whose answers corresponded to the predictions made by basic/ contraposition. Almost none of their conclusions corresponded to predictions made by abduction/ generalization. Theses participants did change their answers from week 1 to week 2 
 lower than average and concluded higher than average NVC. Generally, their answers improved from week 1 to week 2 with respect to CL.
 \item[Intuitive cluster] (7 out of 20 participants)
is characterized by the participants whose answers corresponded to the predictions given by abduction/ generalization. Almost none of their conclusions corresponded to the predictions made by contraposition (except of 2 participants in week 2). They showed a high change within their own answers from week 1 to week 2 
 and concluded lower than average NVC. Generally, their answers improved from week 1 to week 2 with respect to CL, but they never performed \textit{as well as} the participants in 
 the deliberative cluster.
 \item[Intermediate cluster] (7 out of 20 participants) is characterized by the participants whose answers did not corresponded to any particular principle. These participants had a high change within their own answers from week 1 to week 2
 and no particular pattern with respect to NVC conclusions. Generally they were not improving their answers from week 1 to week 2 with respect to CL, but normally compensated for answers valid in CL in some cases (that corresponded to contraposition) with answers not valid in CL in the other cases (that correspond to abduction/ generalization).
\end{description}

 \section{Conclusions and Future Directions}
 
 
We have investigated the cognitive adequacy of clustering by principles under WCS with respect to the individual answer patterns reported by Johnson-Laird and Khemlani~\cite{khemlani:2016}.
Their approach is the only one we are aware of that has evaluated predictions on human syllogistic reasoning
with respect to the non-aggregated data.
However in contrast to here, they did not distinguish between the participants' responses in week 1 and week 2.
 
Our results are two-fold: First, we were able to show that the reasoning clusters as specified in~\cite{declare:2017}, could not be validated
with respect to these individual reasoning patterns. However, and this leads us to the second more interesting result,
we could show that participants' answers seemed to be explainable through a set of cognitive principles. In particular,
some principles seemed to exclude each other by the individual reasoners throughout the experiment (contraposition and abduction), whereas others often appeared together (basic/ contraposition and abduction/ generalization). 
The individual participants' data allowed us to observe other reasoning patterns (high/ low frequency of NVC conclusions and high/ low frequency of changing answers between week 1 and week 2 answers). 
Interestingly, there was one group of participants for which we could not identify any pattern (intermediate)
and might have changed their strategies within the task.
Furthermore, it is quite likely that the set of cognitive principles presented in Section~\ref{sect:representing} is not complete.

% Static clusters for each participant might not be appropriate, rather, a sequential application of cognitive principles seems to be more likely
For the future, we need to investigate whether the sequential application of the cognitive principles 
can give us more insights on the participants' reasoning pattern. In particular, our hypothesis is that these principles might correspond to some of the reasoning steps within the insight model
by Johnson-Laird and Wason~\cite{johnsonlaird:1970} for the selection task~\cite{wason:68}. Similarly, a meta-analysis of the individual response patterns for the selection task has been suggested by Ragni, Kola and Johnson-Laird~\cite{ragni:2017}.
This in turn might give us a new starting point to investigate whether the cognitive principles  specified here can characterize participants' conclusions in other reasoning episodes.


% episodes.
% In order to validate the idea of a sequential reasoning procedure, cognitive scientists will have to set-up new 
% experiments, where possibly, participants are explicitly asked (not) to assume certain cognitive principles.
% Another aspect that needs to be examined, is 


\bibliographystyle{splncs}
\bibliography{bibv4}

\end{document}

\begin{table}[htp]\tiny\centering
\newcolumntype{a}{>{\columncolor{white}}c@{\hspace{2mm}}}
\newcolumntype{b}{>{\columncolor{gray!40}}c@{\hspace{2mm}}}  
\captionsetup{font=footnotesize}
 \begin{tabular}{llllllllll}
 \toprule
	 & \MA ac & \ME ac & \MI ac & \MO ac & \MA ca & \ME ca & \MI ca & \MO ca & \NVC \\ \midrule 
	 
\rowcolor{lightgray} 
\MA\MA1  & \basicc&        &        &        &        &        &        &        &      \\ \rowcolor{lightgray} 
\MA\MA2  &        &        &        &        & \basicc&        &        &        & \\ \rowcolor{lightgray} 
\MA\MA3  &        &        &        &        &        &        &        &        & \basicc \\ \rowcolor{lightgray} 
\MA\MA4  & \basicc&        &        &        &\basicc &        &        &        & \\ %\midrule 

\MA\MI1  &        &        & \abdc  &        &        &        & \abdc  &        & \basicc \\
\MA\MI2  &        &        & \basicc&        &        &        & \basicc \\
\MA\MI3  &        &        & \abdc  &        &        &        & \abdc  &        & \basicc \\
\MA\MI4  &        &        & \basicc&        &        &        & \basicc&        & \\%\midrule 

\rowcolor{lightgray} 
\MA\ME1  &        & \basicc&        &        &        &        &        &        & \\ \rowcolor{lightgray} 
\MA\ME2  &        &\contrac&        &        &        & \basicc&        &        & \\ \rowcolor{lightgray} 
\MA\ME3  &        &\contrac&        &        &        &\contrac&        &        & \basicc \\ \rowcolor{lightgray} 
\MA\ME4  &        & \basicc&        &        &        &\contrac&        &        &  \\%\midrule

\MA\MO1  &        &        &        &\abdc   &        &        &        &        & \basicc \\
\MA\MO2  &        &        &        &        &        &        &        &\basicc \\
\MA\MO3  &        &        &        &        &        &        &\contrac&\basicc \\
\MA\MO4  &        &        &        &\basicc \\%\midrule 

\rowcolor{lightgray} 
\MI\MA1  &        &        &\basicc &        &        &        &\basicc &        & \\\rowcolor{lightgray} 
\MI\MA2  &        &        &\abdc   &        &\genc   &        & \abdc  &        & \basicc \\ \rowcolor{lightgray} 
\MI\MA3  &        &        & \abdc  &        &        &        & \abdc  &        & \basicc \\ \rowcolor{lightgray} 
\MI\MA4  &        &        &\basicc &        &        &        &\basicc &        &  \\  %\midrule

\MI\MI1  &        &        &\genc   &        &        &        & \genc  &        & \basicc \\
\MI\MI2  &        &        &\genc   &        &        &        & \genc  &        & \basicc \\
\MI\MI3  &        &        &\genc   &        &        &        & \genc  &        & \basicc \\
\MI\MI4  &        &        &\genc   &        &        &        & \genc  &        & \basicc \\ %\midrule 

\rowcolor{lightgray} 
\MI\ME1  &        &        &        &\basicc &        &        &        &        & \\\rowcolor{lightgray} 
\MI\ME2  &        &        &        &\contrac&        & \genc  &        &        & \basicc \\\rowcolor{lightgray} 
\MI\ME3  &        &        &        &\contrac&        & \genc  &        &        & \basicc  \\\rowcolor{lightgray} 
\MI\ME4  &        &        &        & \basicc&        &        &        &        & \\ %\midrule 

\MI\MO1  &        &        &        & \genc  &        &        &        &        & \basicc \\
\MI\MO2  &        &        &        & \genc  &        &        &        &        & \basicc \\ 
\MI\MO3  &        &        &        & \genc  &        &        &        &        & \basicc \\ 
\MI\MO4  &        &        &        & \genc  &        &        &        &        & \basicc \\%\midrule

\rowcolor{lightgray} 
\ME\MA1  &        & \basicc&        &        &        &        &        &        & \\\rowcolor{lightgray} 
\ME\MA2  &        &\contrac&        &        &        & \basicc&        &        &  \\\rowcolor{lightgray} 
\ME\MA3  &        &\contrac&        &        &        &\contrac&        &        & \basicc \\\rowcolor{lightgray} 
\ME\MA4  &        &\contrac&        &        &        & \basicc&        &        &\\  %\midrule 

\ME\MI1  &        & \genc   &        &	 &	  &	   &	    &	     & \basicc \\
\ME\MI2  &        &	       &	&	 &	  &	   &	    & \basicc \\
\ME\MI3  &        & \genc   &        &	 &	  &	   &	    &\contrac& \basicc \\
\ME\MI4  &        &	       &	&	 &	  &	   &	    & \basicc \\%\midrule

\rowcolor{lightgray} 
\ME\ME1  &        &	       &	&	 &	  &	   &	    &        &\basicc \\ \rowcolor{lightgray} 
\ME\ME2  &        &	       &	&	 &	  &	   &	    &        &\basicc \\ \rowcolor{lightgray} 
\ME\ME3  &        &	       &	&	 &	  &	   &	    &        &\basicc \\ \rowcolor{lightgray} 
\ME\ME4  &        &	       &	&	 &	  &	   &	    &        &\basicc \\%\midrule

\ME\MO1  &        &	       &	&	 &	  &	   &	    &        &\basicc \\
\ME\MO2  &        &	       &	&	 &	  &	   &	    &        &\basicc \\
\ME\MO3  &        &	       &	&	 &	  &	   &	    &        &\basicc \\
\ME\MO4  &        &	       &	&	 &	  &	   &	    &        &\basicc \\ %\midrule

\rowcolor{lightgray} 
\MO\MA1  &        &            &	& \basicc&        &        &        &        & \\\rowcolor{lightgray} 
\MO\MA2  &        &	       &	&	 &	  &	   &	    &  \abdc &\basicc \\\rowcolor{lightgray} 
\MO\MA3  &        &	       &	&\contrac&	  &	   &	    &        &\basicc \\\rowcolor{lightgray} 
\MO\MA4  &        &	       &	&	 &	  &	   &	    & \basicc& \\  %\midrule

\MO\MI1  &        &	       &	&\genc   &	  &	   &	    &        &\basicc \\
\MO\MI2  &        &	       &	&	 &	  &	   &	    &\genc   &\basicc \\
\MO\MI3  &        &	       &	&\genc   &	  &	   &	    &        &\basicc \\
\MO\MI4  &        &	       &	&	 &	  &	   &	    &\genc   &\basicc\\ %\midrule 

\rowcolor{lightgray} 
\MO\ME1  &        &	       &	&        &	  &	   &	    &        &\basicc \\\rowcolor{lightgray} 
\MO\ME2  &        &	       &	&        &	  &	   &	    &        &\basicc \\\rowcolor{lightgray} 
\MO\ME3  &        &	       &	&        &	  &	   &	    &        &\basicc \\\rowcolor{lightgray} 
\MO\ME4  &        &	       &	&        &	  &	   &	    &        &\basicc \\ %\midrule

\MO\MO1  &        &	       &	&        &	  &	   &	    &        &\basicc \\
\MO\MO2  &        &	       &	&        &	  &	   &	    &        &\basicc \\
\MO\MO3  &        &	       &	&        &	  &	   &	    &        &\basicc \\
\MO\MO4  &        &	       &	&        &	  &	   &	    &        &\basicc \\ \bottomrule
  \end{tabular}

\caption{Predictions according to the cognitive principles as specified in Section~\ref{sect:representing}.\betweentinyandsmall}
 \end{table}
 
 

% \commentem{Dieser Table ist falsch!!!}
% \begin{table}
% \begin{center}
% \begin{adjustbox}{totalheight={20.5cm}} 
% 
% \begin{tabular}{c@{\hspace{3cm}}c}
% 
%   \begin{tabular}{r|c|c|c|c|c|c|c|c|c|} 
%  Premises & Aac & Iac & Eac & Oac & Aca & Ica & Eca & Oca & NVC \\ 
%  \cline{2-10} 
%  AA1 &\cellcolor{black} & & & & & & & & \\ 
%  \cline{2-10} 
%  AA2 &&&&&\cellcolor{black} & &&&\\ 
%  \cline{2-10} 
%  AA3 &&\cellcolor{gray!50} & &&\cellcolor{gray!50} & &&&\cellcolor{gray!100}  \\ 
%  \cline{2-10} 
%  AA4 &\cellcolor{gray!50} & \cellcolor{gray!50} & &&&&&&\cellcolor{gray!50}  \\ 
%  \cline{2-10} 
%  AE1 &&&\cellcolor{black} & &&&&&\\ 
%  \cline{2-10} 
%  AE2 &&&&&&&\cellcolor{gray!100} & &\cellcolor{gray!50}  \\ 
%  \cline{2-10} 
%  AE3 &&&\cellcolor{gray!50} & &&&\cellcolor{gray!100} & &\\ 
%  \cline{2-10} 
%  AE4 &&&\cellcolor{gray!50} & \cellcolor{gray!50} & &&&&\cellcolor{gray!50} \\ 
%  \cline{2-10} 
%  AI1 &&\cellcolor{black} & &&&&&&\cellcolor{gray!50} \\ 
%  \cline{2-10} 
%  AI2 &&&&&&\cellcolor{black} & &&\\ 
%  \cline{2-10} 
%  AI3 &&&&&&\cellcolor{gray!50} & &&\cellcolor{black}  \\ 
%  \cline{2-10} 
%  AI4 &&\cellcolor{gray!100} & &&&\cellcolor{gray!100} & &&\\ 
%  \cline{2-10} 
%  AO1 &&&&\cellcolor{black} & &&&&\cellcolor{gray!50}  \\ 
%  \cline{2-10} 
%  AO2 &&&&&&&&\cellcolor{gray!100} & \cellcolor{gray!100} \\ 
%  \cline{2-10} 
%  AO3 &&&&&&&&\cellcolor{gray!100} & \cellcolor{gray!50}  \\ 
%  \cline{2-10} 
%  AO4 &&&&\cellcolor{black} & &&&&\\ 
%  \cline{2-10} 
%  EA1 &&&\cellcolor{gray!100} & &&&\cellcolor{gray!50} & &\cellcolor{gray!50}  \\ 
%  \cline{2-10} 
%  EA2 &&&&&&&\cellcolor{black} & &\\ 
%  \cline{2-10} 
%  EA3 &&&\cellcolor{gray!100} & &&&\cellcolor{gray!100} & &\\ 
%  \cline{2-10} 
%  EA4 &&&\cellcolor{gray!50} & &&&\cellcolor{gray!50} & \cellcolor{gray!50} & \cellcolor{gray!50}  \\ 
%  \cline{2-10} 
%  EE1 &&&\cellcolor{gray!50} & &&&&&\cellcolor{gray!100}  \\ 
%  \cline{2-10} 
%  EE2 &&&&&&&\cellcolor{gray!50} & &\cellcolor{black}  \\ 
%  \cline{2-10} 
%  EE3 &&&&&&&&&\cellcolor{black}  \\ 
%  \cline{2-10} 
%  EE4 &&&&&&&&&\cellcolor{black}  \\ 
%  \cline{2-10} 
%  EI1 &&&\cellcolor{gray!50} & &&&&\cellcolor{gray!50} & \cellcolor{gray!50}  \\ 
%  \cline{2-10} 
%  EI2 &&&&&&&&\cellcolor{black} & \cellcolor{gray!50}  \\ 
%  \cline{2-10} 
%  EI3 &&&&&&&&\cellcolor{gray!100} & \cellcolor{gray!50}  \\ 
%  \cline{2-10} 
%  EI4 &&&&&&&&\cellcolor{gray!100} & \cellcolor{gray!50}  \\ 
%  \cline{2-10} 
%  EO1 &&&&\cellcolor{gray!50} & &&&&\cellcolor{black}  \\ 
%  \cline{2-10} 
%  EO2 &&&&&&&&\cellcolor{gray!50} & \cellcolor{black} \\ 
%  \cline{2-10} 
%  EO3 &&&&&&&&&\cellcolor{black}  \\ 
%  \cline{2-10} 
%  EO4 &&&&\cellcolor{gray!50} & &&&&\cellcolor{black}  \\ 
%  \cline{2-10} 
%  IA1 &&\cellcolor{black} & &&&&&&\\ 
%  \cline{2-10} 
%  IA2 &&&&&&\cellcolor{black} & &&\cellcolor{gray!50}  \\ 
%  \cline{2-10} 
%  IA3 &&\cellcolor{gray!50} & &&&\cellcolor{gray!50} & &&\cellcolor{gray!100}  \\ 
%  \cline{2-10} 
%  IA4 &&\cellcolor{gray!100} & &&&\cellcolor{gray!100} & &&\\ 
%  \cline{2-10} 
%  IE1 &&&&\cellcolor{black} & &&&&\\ 
%  \cline{2-10} 
%  IE2 &&&&\cellcolor{gray!50} & &&\cellcolor{gray!50} & &\cellcolor{gray!100}  \\ 
%  \cline{2-10} 
%  IE3 &&&\cellcolor{gray!50} & \cellcolor{gray!50} & &&\cellcolor{gray!50} & &\cellcolor{gray!50} \\ 
%  \cline{2-10} 
%  IE4 &&&&\cellcolor{gray!100} & &&&&\cellcolor{gray!50}  \\ 
%  \cline{2-10} 
%  II1 &&\cellcolor{gray!50} & &&&&&&\cellcolor{black}  \\ 
%  \cline{2-10} 
%  II2 &&&&&&\cellcolor{gray!50} & &&\cellcolor{black} \\ 
%  \cline{2-10} 
%  II3 &&\cellcolor{gray!50} & &&&&&&\cellcolor{black}  \\ 
%  \cline{2-10} 
%  II4 &&&&&&&&&\cellcolor{black}  \\ 
%  \cline{2-10} 
%  IO1 &&&&\cellcolor{gray!50} & &&&&\cellcolor{black}  \\ 
%  \cline{2-10} 
%  IO2 &&&&&&&&&\cellcolor{black}  \\ 
%  \cline{2-10} 
%  IO3 &&&&&&&&&\cellcolor{black}  \\ 
%  \cline{2-10} 
%  IO4 &&&&&&&&&\cellcolor{black}  \\ 
%  \cline{2-10} 
%  OA1 &&&&\cellcolor{gray!50} & &&&&\cellcolor{gray!100}  \\ 
%  \cline{2-10} 
%  OA2 &&&&&&&&\cellcolor{black} & \cellcolor{gray!50}  \\ 
%  \cline{2-10} 
%  OA3 &&&&\cellcolor{gray!50} & &&&\cellcolor{gray!50} & \cellcolor{gray!100}  \\ 
%  \cline{2-10} 
%  OA4 &&&&&&&&\cellcolor{black} & \cellcolor{gray!50}  \\ 
%  \cline{2-10} 
%  OE1 &&&&\cellcolor{gray!50} & &&&&\cellcolor{black} \\ 
%  \cline{2-10} 
%  OE2 &&&&&&&&\cellcolor{gray!50} & \cellcolor{black} \\ 
%  \cline{2-10} 
%  OE3 &&&&&&&&&\cellcolor{black} \\ 
%  \cline{2-10} 
%  OE4 &&&&&&&&&\cellcolor{black}  \\ 
%  \cline{2-10} 
%  OI1 &&&&\cellcolor{gray!50} & &&&&\cellcolor{black}  \\ 
%  \cline{2-10} 
%  OI2 &&&&&&&&\cellcolor{gray!50} & \cellcolor{black}  \\ 
%  \cline{2-10} 
%  OI3 &&&&&&&&&\cellcolor{black}  \\ 
%  \cline{2-10} 
%  OI4 &&&&&&&&&\cellcolor{black}  \\ 
%  \cline{2-10} 
%  OO1 &&&&&&&&&\cellcolor{black} \\ 
%  \cline{2-10} 
%  OO2 &&&&&&&&&\cellcolor{black}  \\ 
%  \cline{2-10} 
%  OO3 &&&&&&&&&\cellcolor{black}  \\ 
%  \cline{2-10} 
%  OO4 &&&&&&&&&\cellcolor{black}  \\ 
%  \cline{2-10} 
%   \end{tabular}
%   & 
%  
%   \begin{tabular}{r|c|c|c|c|c|c|c|c|c|} 
%  Premises & Aac & Iac & Eac & Oac & Aca & Ica & Eca & Oca & NVC \\ 
%  \cline{2-10} 
%  AA1 &\cellcolor{gray!100} & &&&&&&&\cellcolor{gray!50}  \\ 
%  \cline{2-10} 
%  AA2 &&&&&\cellcolor{gray!100} & &&&\cellcolor{gray!50}  \\ 
%  \cline{2-10} 
%  AA3 &&&&&&&&&\cellcolor{black} \\ 
%  \cline{2-10} 
%  AA4 &\cellcolor{gray!50} & &&&\cellcolor{gray!50} & &&&\cellcolor{gray!50}  \\ 
%  \cline{2-10} 
%  AE1 &&&\cellcolor{gray!50} & &&&\cellcolor{gray!50} & &\cellcolor{gray!50}  \\ 
%  \cline{2-10} 
%  AE2 &&&\cellcolor{gray!50} & &&&\cellcolor{gray!50} & &\cellcolor{gray!50}  \\ 
%  \cline{2-10} 
%  AE3 &&&\cellcolor{gray!50} & &&&\cellcolor{gray!50} & &\cellcolor{gray!50}  \\ 
%  \cline{2-10} 
%  AE4 &&&\cellcolor{gray!50} & &&&\cellcolor{gray!50} & &\cellcolor{gray!50}  \\ 
%  \cline{2-10} 
%  AI1 &&\cellcolor{gray!100} & &&&&&&\cellcolor{gray!50}  \\ 
%  \cline{2-10} 
%  AI2 &&&&&&\cellcolor{gray!100} & &&\cellcolor{gray!50}  \\ 
%  \cline{2-10} 
%  AI3 &&&&&&&&&\cellcolor{black} \\ 
%  \cline{2-10} 
%  AI4 &&\cellcolor{gray!100} & &&&&&&\cellcolor{gray!50}  \\ 
%  \cline{2-10} 
%  AO1 &&&&\cellcolor{gray!100} & &&&&\cellcolor{gray!50}  \\ 
%  \cline{2-10} 
%  AO2 &&&&&&&&\cellcolor{black} & \cellcolor{gray!50}  \\ 
%  \cline{2-10} 
%  AO3 &&&&&&&&\cellcolor{gray!50} & \cellcolor{black}  \\ 
%  \cline{2-10} 
%  AO4 &&&&\cellcolor{black} & &&&&\cellcolor{gray!50} \\ 
%  \cline{2-10} 
%  EA1 &&&\cellcolor{gray!50} & &&&\cellcolor{gray!50} & &\cellcolor{gray!50}  \\ 
%  \cline{2-10} 
%  EA2 &&&\cellcolor{gray!50} & &&&\cellcolor{gray!50} & &\cellcolor{gray!50}  \\ 
%  \cline{2-10} 
%  EA3 &&&\cellcolor{gray!50} & &&&\cellcolor{gray!50} & &\cellcolor{gray!50}  \\ 
%  \cline{2-10} 
%  EA4 &&&\cellcolor{gray!50} & &&&\cellcolor{gray!50} & &\cellcolor{gray!50}  \\ 
%  \cline{2-10} 
%  EE1 &&&&&&&&&\cellcolor{black}  \\ 
%  \cline{2-10} 
%  EE2 &&&&&&&&&\cellcolor{black}  \\ 
%  \cline{2-10} 
%  EE3 &&&&&&&&&\cellcolor{black}  \\ 
%  \cline{2-10} 
%  EE4 &&&&&&&&&\cellcolor{black}  \\ 
%  \cline{2-10} 
%  EI1 &&&&\cellcolor{black} & &&&&\cellcolor{gray!50}  \\ 
%  \cline{2-10} 
%  EI2 &&&&\cellcolor{gray!50} & &&&&\cellcolor{gray!100}  \\ 
%  \cline{2-10} 
%  EI3 &&&&\cellcolor{gray!50} & &&&&\cellcolor{gray!100}  \\ 
%  \cline{2-10} 
%  EI4 &&&&\cellcolor{black} & &&&&\cellcolor{gray!50}  \\ 
%  \cline{2-10} 
%  EO1 &&&&&&&&&\cellcolor{black}  \\ 
%  \cline{2-10} 
%  EO2 &&&&&&&&&\cellcolor{black} \\ 
%  \cline{2-10} 
%  EO3 &&&&&&&&&\cellcolor{black}  \\ 
%  \cline{2-10} 
%  EO4 &&&&&&&&&\cellcolor{black}  \\ 
%  \cline{2-10} 
%  IA1 &&\cellcolor{gray!100} & &&&&&&\cellcolor{gray!50}  \\ 
%  \cline{2-10} 
%  IA2 &&&&&&\cellcolor{gray!100} & &&\cellcolor{gray!50}  \\ 
%  \cline{2-10} 
%  IA3 &&&&&&&&&\cellcolor{black}  \\ 
%  \cline{2-10} 
%  IA4 &&&&&&\cellcolor{gray!100} & &&\cellcolor{gray!50}  \\ 
%  \cline{2-10} 
%  IE1 &&&&&&&&&\cellcolor{black} \\ 
%  \cline{2-10} 
%  IE2 &&&&&&&&\cellcolor{black} & \cellcolor{gray!50}  \\ 
%  \cline{2-10} 
%  IE3 &&&&&&&&\cellcolor{gray!50} & \cellcolor{gray!100}  \\ 
%  \cline{2-10} 
%  IE4 &&&&&&&&\cellcolor{black} & \cellcolor{gray!50}  \\ 
%  \cline{2-10} 
%  II1 &&&&&&&&&\cellcolor{black}  \\ 
%  \cline{2-10} 
%  II2 &&&&&&&&&\cellcolor{black}  \\ 
%  \cline{2-10} 
%  II3 &&&&&&&&&\cellcolor{black}  \\ 
%  \cline{2-10} 
%  II4 &&&&&&&&&\cellcolor{black}  \\ 
%  \cline{2-10} 
%  IO1 &&&&&&&&&\cellcolor{black}  \\ 
%  \cline{2-10} 
%  IO2 &&&&&&&&&\cellcolor{black}  \\ 
%  \cline{2-10} 
%  IO3 &&&&&&&&&\cellcolor{black}  \\ 
%  \cline{2-10} 
%  IO4 &&&&&&&&&\cellcolor{black}  \\ 
%  \cline{2-10} 
%  OA1 &&&&\cellcolor{black} & &&&&\cellcolor{gray!50}  \\ 
%  \cline{2-10} 
%  OA2 &&&&&&&&\cellcolor{gray!100} & \cellcolor{gray!50}  \\ 
%  \cline{2-10} 
%  OA3 &&&&\cellcolor{gray!50} & &&&&\cellcolor{black}  \\ 
%  \cline{2-10} 
%  OA4 &&&&&&&&\cellcolor{black} & \cellcolor{gray!50}  \\ 
%  \cline{2-10} 
%  OE1 &&&&&&&&&\cellcolor{black}  \\ 
%  \cline{2-10} 
%  OE2 &&&&&&&&&\cellcolor{black}  \\ 
%  \cline{2-10} 
%  OE3 &&&&&&&&&\cellcolor{black}  \\ 
%  \cline{2-10} 
%  OE4 &&&&&&&&&\cellcolor{black}  \\ 
%  \cline{2-10} 
%  OI1 &&&&&&&&&\cellcolor{black}  \\ 
%  \cline{2-10} 
%  OI2 &&&&&&&&&\cellcolor{black}  \\ 
%  \cline{2-10} 
%  OI3 &&&&&&&&&\cellcolor{black}  \\ 
%  \cline{2-10} 
%  OI4 &&&&&&&&&\cellcolor{black}  \\ 
%  \cline{2-10} 
%  OO1 &&&&&&&&&\cellcolor{black}  \\ 
%  \cline{2-10} 
%  OO2 &&&&&&&&&\cellcolor{black} \\ 
%  \cline{2-10} 
%  OO3 &&&&&&&&&\cellcolor{black}  \\ 
%  \cline{2-10} 
%  OO4 &&&&&&&&&\cellcolor{black}  \\ 
%  \cline{2-10} 
%   \end{tabular} 
% \end{tabular}
% \end{adjustbox} 
% \end{center}
% \caption{The data reported in~\cite{khemlani:2012} (left) 
% and the predictions made by WCS Clustering (right):
% White means $< 16\%$, {\color{gray!50}light gray means $< 40\%$}, {\color{gray!100}gray means $< 60\%$}
% and {black means $>60\%$}.
%  \label{tab:eval:aggregated}}
% \end{table}
% Table~\ref{tab:eval:aggregated} gives an overview of the participants conclusions reported in~\cite{khemlani:2012}
% and the predictions under the WCS clustering. 
